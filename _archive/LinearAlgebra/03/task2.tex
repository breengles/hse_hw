%!TEX root = kotov.tex
\section{Task 2}
\begin{task}
    Вычислите определитель порядка $2n$
    \begin{eqnarray}
        \begin{Vmatrix}
            a & 0 & 0 & \ldots & 0 & 0 & b \\
            0 & a & 0 & \ldots & 0 & b & 0 \\
            0 & 0 & a & \ldots & b & 0 & 0 \\
              &   &   & \vdots &   &   &   \\
            0 & 0 & b & \ldots & a & 0 & 0 \\
            0 & b & 0 & \ldots & 0 & a & 0 \\
            b & 0 & 0 & \ldots & 0 & 0 & a
        \end{Vmatrix}
    \end{eqnarray}
\end{task}

\begin{solution}
    Рассмотрим разложение по первой строке:
    \begin{gather}
        \mathcal{D}_{2n} = 
        \begin{Vmatrix}
            a & 0 & 0 & \ldots & 0 & 0 & b \\
            0 & a & 0 & \ldots & 0 & b & 0 \\
            0 & 0 & a & \ldots & b & 0 & 0 \\
              &   &   & \vdots &   &   &   \\
            0 & 0 & b & \ldots & a & 0 & 0 \\
            0 & b & 0 & \ldots & 0 & a & 0 \\
            b & 0 & 0 & \ldots & 0 & 0 & a
        \end{Vmatrix} = a^2(-1)^{2n-2}
        \underbrace{
        \begin{Vmatrix}
            a & 0 & 0 & \ldots & 0 & 0 & b \\
            0 & a & 0 & \ldots & 0 & b & 0 \\
            0 & 0 & a & \ldots & b & 0 & 0 \\
              &   &   & \vdots &   &   &   \\
            0 & 0 & b & \ldots & a & 0 & 0 \\
            0 & b & 0 & \ldots & 0 & a & 0 \\
            b & 0 & 0 & \ldots & 0 & 0 & a
        \end{Vmatrix}}_{2(n-1)} - b^2(-1)^{2n-2}
        \underbrace{
        \begin{Vmatrix}
            a & 0 & 0 & \ldots & 0 & 0 & b \\
            0 & a & 0 & \ldots & 0 & b & 0 \\
            0 & 0 & a & \ldots & b & 0 & 0 \\
              &   &   & \vdots &   &   &   \\
            0 & 0 & b & \ldots & a & 0 & 0 \\
            0 & b & 0 & \ldots & 0 & a & 0 \\
            b & 0 & 0 & \ldots & 0 & 0 & a
        \end{Vmatrix}}_{2(n-1)}
    \end{gather}
    \begin{remark}
        $(-1)^{2n-2}$ вылезают от соответствующих алгебраических дополнений, то есть от матриц, где в последней строке все элементы равны 0, кроме последнего для $a$ и первого для $b$, то есть от определителей вида
        \begin{equation}
            \begin{Vmatrix}
                a & 0 & \ldots & 0 & b & 0 \\
                0 & a & \ldots & b & 0 & 0 \\
                & &\vdots \\
                0 & & \ldots & & 0 & a
            \end{Vmatrix},
        \end{equation}
        разложенных по последней строке (аналогично для $b$).
    \end{remark}
    Заметим, что размер матрицы справа при $a^2$ и $b^2$ на $2$ меньше, чем размер исходной, а вид у нее такой же, то есть можно написать рекуррентную формулу $\mathcal{D}_{2n} = (a^2 - b^2)\mathcal{D}_{2n-2}$, стандартно с помощью характеристического уравнения находим решение для рекуррента в явном виде при начальном условии $\mathcal{D}_2 = a^2 - b^2$: $\mathcal{D}_{2n} = (a^2 - b^2)^n$
\end{solution}
\begin{remark}
    Можно решить намного короче, если прибавить первые $n$ столбцов к последним (то есть к $n$-ому прибавить первый, к $n-1$-ому прибавить $2$-ой и т.д.), а затем вычесть из первых $n$ строк последние по такой же схеме.
    Тогда получится нижнетреугольная матрица, в которой на диагонали стоят $n$ штук $(a-b)$ и $n$ штук $(a+b)$.
    Определитель такой матрицы $\mathcal{D} = (a-b)^n(a+b)^n = (a^2 - b^2)^n$
\end{remark}