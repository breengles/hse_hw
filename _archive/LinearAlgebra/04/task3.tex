%!TEX root = kotov.tex
\section{Task 3}
\begin{task}
    $a:V\longrightarrow V$ оператор. Найдите базис циклического подпространства, порожденного вектором $v$ и матрицу сужения $a$ на это подпространство в найденном базисе. $$V=M_2(\mathbb{R}),\, a(x)=x+x^T,\, v=\begin{pmatrix}1&1\\-1&1\end{pmatrix}$$
\end{task}

\begin{solution}
    Пусть $\vec{v}_1 = v = 
    \begin{pmatrix}
        1 & 1 \\
        -1 & 1
    \end{pmatrix}$, тогда будем строить циклическую группу:
    \begin{gather}
        \vec{v}_1 = 
        \begin{pmatrix}
             1 & 1 \\
            -1 & 1
        \end{pmatrix} \\
        \vec{v}_2 = a\vec{v}_1 = 
        \begin{pmatrix}
            2 & 0 \\ 0 & 2
        \end{pmatrix} \\
        \vec{v}_3 = a\vec{v}_2 = 
        \begin{pmatrix}
            4 & 0 \\ 0 & 4
        \end{pmatrix} = 2\vec{v}_2
    \end{gather}
    То есть на третьем элементе мы уже получили линейно-зависимый набор, тогда
    \begin{eqnarray}
        \left<
            \begin{pmatrix}
                1 & 1 \\ -1 & 1
            \end{pmatrix},
            \begin{pmatrix}
                2 & 0 \\
                0 & 2
            \end{pmatrix}
        \right> \text{--- циклическое подпространство, порожденное } \vec{v}_1.
    \end{eqnarray}

    Теперь рассмотрим матрицу $a$ в базисе этого подпространства:
    \begin{gather}
        \begin{rcases}
            a\vec{v}_1 = \vec{v}_2 \\
            a\vec{v}_2 = 2\vec{v}_2 
        \end{rcases} \Longrightarrow a =
        \begin{pmatrix}
            0 & 1 \\
            0 & 2
        \end{pmatrix}
    \end{gather}

    Проверка:
    \begin{eqnarray}
        \begin{pmatrix}
            0 & 1 \\ 0 & 2
        \end{pmatrix}
        \begin{pmatrix}
            \vec{v}_1 \\ \vec{v}_2
        \end{pmatrix} = 
        \begin{pmatrix}
            \vec{v}_2 \\ 2\vec{v}_2
        \end{pmatrix}
    \end{eqnarray}
\end{solution}