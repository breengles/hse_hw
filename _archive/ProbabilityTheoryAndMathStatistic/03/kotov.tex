%!TEX program = lualatex
\documentclass[a4paper,12pt]{article}
\usepackage{settings}
\usepackage{cancel}
\usepackage{ulem}

\newenvironment{task}{
    \par\noindent\textbf{Условие:}
}

\newenvironment{upd}
    {\par\noindent\textbf{Update:}}
    {\par\noindent\rule{\textwidth}{1pt}}

\newenvironment{solution}{
\begin{proof}[Решение]
    \mbox{}\par}
{\end{proof}}

\makeatletter
\newcommand\mathcircled[1]{%
  \mathpalette\@mathcircled{#1}%
}
\newcommand\@mathcircled[2]{%
  \tikz[baseline=(math.base)] \node[draw,circle,inner sep=1pt] (math) {$\m@th#1#2$};%
}
\makeatother

\setcounter{secnumdepth}{0}

\hypersetup{
    unicode=true,  % non-Latin characters in Acrobat’s bookmarks
    pdffitwindow=true,  % window fit to page when opened
    pdfstartview={FitH},  % fits the width of the page to the window
    pdfauthor={Котов А. А.},  % author
    pdfnewwindow=true,  % links in new PDF window
    colorlinks=true,  % false: boxed links; true: colored links
    linkcolor=linkcolor,  % color of internal links (change box color with linkbordercolor)
    citecolor=citecolor,  % color of links to bibliography
    filecolor=magenta,  % color of file links
    urlcolor=urlcolor  % color of external links
}

\definecolor{dkgreen}{rgb}{0,0.6,0}
\definecolor{gray}{rgb}{0.5,0.5,0.5}
\definecolor{mauve}{rgb}{0.58,0,0.82}  

\lstset{ %
language=python,                % the language of the code
basicstyle=\footnotesize,           % the size of the fonts that are used for the code
numbers=left,                   % where to put the line-numbers
numberstyle=\tiny\color{gray},  % the style that is used for the line-numbers
stepnumber=1,                   % the step between two line-numbers. If it's 1, each line 
                                % will be numbered
numbersep=5pt,                  % how far the line-numbers are from the code
backgroundcolor=\color{white},      % choose the background color. You must add \usepackage{color}
showspaces=false,               % show spaces adding particular underscores
showstringspaces=false,         % underline spaces within strings
showtabs=true,                 % show tabs within strings adding particular underscores
frame=single,                   % adds a frame around the code
rulecolor=\color{black!10},        % if not set, the frame-color may be changed on line-breaks within not-black text (e.g. comments (green here))
tabsize=4,                      % sets default tabsize to 2 spaces
captionpos=b,                   % sets the caption-position to bottom
breaklines=true,                % sets automatic line breaking
breakatwhitespace=false,        % sets if automatic breaks should only happen at whitespace
title=\lstname,                   % show the filename of files included with \lstinputlisting;
                                % also try caption instead of title
keywordstyle=\color{blue},          % keyword style
commentstyle=\color{dkgreen},       % comment style
stringstyle=\color{mauve},        % string literal style
escapeinside={\%*}{*)},            % if you want to add LaTeX within your code
morekeywords={done, to},              % if you want to add more keywords to the set
%  deletekeywords={...}              % if you want to delete keywords from the given language
}


\author{Котов Артем, МОиАД2020}
\title{Домашняя работа}
\date{\today}

\begin{document}
    \maketitle
    % \tableofcontents
    % \newpage

    %!TEX root = artem_kotov_v1.tex
\section{Task 1}

\begin{task}
    Вычисляем $p(a), p(b), p(c), p(d), p(c|a), p(c|b), p(c|d), p(c|ab), p(c|abd)$
\end{task}

\begin{solution}
    \begin{remark}
        $U[a,b] \sim \frac{1}{b - a + 1}[a \le x \le b]$
        $Bin(n, p) \sim {n \choose x} p^x (1 - p)^{n - x}$
    \end{remark}

    \begin{gather}
        p(a) = \frac{1}{16}[75 \le a \le 90] \\
        p(b) = \frac{1}{101}[500 \le b \le 600]
    \end{gather}
    Для вычисления $p(c)$ нам потребуется $p(c|ab)$ (для второй модели в целом, аналогично, разве что свертка упростится):
    \begin{gather}
        p_{\text{model}1}(c|ab) \sim \Bin(a, p_1) + \Bin(b, p_2) \sim \sum_i^c {a \choose i} p_1^i (1 - p_1)^{a - i} {b \choose c - i} p_2^{c - i} (1 - p_2)^{b - c + i} \\
        p_{\text{model}2}(c|ab) \sim \Poisson(ap_1 + bp_2)
    \end{gather}
    тогда
    \begin{equation}
        p(c) = \sum_{ab} p(c|ab)p(a)p(b) = \frac{1}{1616} \sum_{a = a_{\min}}^{a_{\max}} \sum_{b = b_{\min}}^{b_{\max}} \sum_{i = 0}^c {a \choose i} {b \choose c - i} p_1^i p_2^{c - i} (1 - p_1)^{a - i} (1 - p_2)^{b - c + i}.
    \end{equation}
    Но это не удобно программировать, лучше оставить в виде сверток двух биномиальных:
    \begin{equation}
        p(c|ab) = \sum_{x = 0}^c p(\Bin(a, p_1) = x) p(\Bin(b, p_2) = c - x)
    \end{equation}

    \begin{gather}
        p(c|a) = \sum_b p(c|ab)p(b) = \frac{1}{101} \sum_{b = 500}^{600} p(c|ab) \\
        p(c|b) = \sum_a p(c|ab)p(a) = \frac{1}{16} \sum_{a = 75}^{90} p(c|ab) \\
        p(c) = \sum_{ab} p(c|ab) \underbrace{p(a)p(b)}_{p(ab)}
    \end{gather}

    Теперь с $d$:
    \begin{gather}
        p(d|c) = p(c + \Bin(c, p_3) = d) = p(\Bin(c, p_3) = d - c) \\
        p(d) = \sum_c p(d|c)p(c) \\
        p(c|d) = \frac{p(d|c) p(c)}{p(d)} \\
        p(c|abd) = \frac{P(abcd)}{p(abd)} = \frac{p(d|c) p(c|ab) p(a) p(b)}{\sum_c p(d|c) p(c|ab) p(a) p(b)}
    \end{gather}
\end{solution}

\end{document}