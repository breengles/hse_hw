%!TEX root = kotov.pdf
\section{Task 4}
\begin{task}
    Про Васю, Петю и вечное желание занять чужое место.
\end{task}

Ох ну и странная, как мне кажется, задачка. Какой-то Xzibit с мониторами получается (простите, не удержался не вставить этот боян).
\begin{solution}
    Рассмотрим пары вершин $(i,j)$ исходного графа. Зададим для каждой такой пары вопрос $f(i,j) \geq d$? Если да, то сопоставим такой \textbf{паре вершин} вершину в новом графе, которую обозначим так же, как и саму пару. Примечательно, что в самом худшем случае новый граф содержит $V^2$ вершин (т.е. вообще все пары вершин доступны, и Петя с Васей могут гулять, где хотят).

    Теперь надо понять, что есть ребра в новом графе. Рассмотрим ребро $e(a, b)$ исходного графа. Тогда, это мы могли бы использовать это ребро (в смысле, кто-то из друзей мог бы по нему гипотетически пройти), то есть в новом графе между вершинами $(a, i)$ и $(b, i)$ (аналогично для $(i, a)$ и $(i, b)$) для всех допустимых (с точки зрения функции $f$) вершин $i$ есть ребро, если в исходном графе есть ребро между $(a, b)$. В самом богатом случае мы бы имели $VE$ ребер (как каждое ребро с каждой вершиной).

    Теперь, когда у нас есть такой граф на графе, нам надо найти самый короткий путь из вершины $(v, p)$ в вершину $(p, v)$. Вроде как, можем просто воспользоваться поиском в ширину от $(v, p)$.
    
    Почему поиск в ширину? Потому что в процессе обхода графа поиск в ширину также вычисляет минимальное количество ребер между входной вершиной и каждой достижимой из нее вершиной, что как раз нам и надо. Это можно интуитивно (строгое доказательство что-то большое, как мне кажется) увидеть через сравнение с поиском в глубину: в \texttt{dfs} мы спускались глубже сразу же, если могли спуститься глубже, а в поиске в ширину мы видим, что можем пойти, но сначала смотрим куда мы еще можем пойти, то есть поиск в глубину мог бы гипотетически поместить в своем дереве конечную вершину в какой-нибудь очень глубокий лист, даже если реально исходная вершина сразу же соединена с ``конечной'', а тут мы увидим, что она рядом и сможем сразу к ней перейти (помахал руками в воздухе).

    По сложности на составление нового графа мы потратим $\O(V^2 + VE)$, а поиск в ширину стоит нам, вообще говоря, столько же (количество вершин плюс количество ребер).
\end{solution}