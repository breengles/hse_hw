%!TEX root = kotov.tex
\section{Task 5}
\begin{task}
    Пусть дан массив размера $n$ из целых чисел. 
    Требуется сделать предварительные вычисления (в будущем мы будем кратко называть их \textit{предподсчёт}) за 
    $\O(n\log n)$, чтобы затем ответить на некоторое неизвестное число запросов про массив вида 
    ``сколько раз число $x$ встречается на отрезке $[l..r]$'', 
    причём на каждый запрос можно потратить $\O(\log n)$ времени.
\end{task}
\begin{solution}
    
    \textit{Предподсчет:} Создадим массив $b_i = (a_i, i)$ и отсортируем его по первому ключу. Затем, в рамках одинакового ключа, отсортируем элементы по второму ключу. Таким образом в массив $b$ будем содержать сортированные элементы исходного массива с запомненным исходными индексами, расположенными по возрастанию. Данные операции выполнятся за $\O(n\log n)$

    Теперь, два раза запускаем бинпоиск в $b$ с данным $x$ с условиями на второй ключ: ``не левее'', чем $l$ для первого бинпоиска (индекс найденного элемента в массиве $b$ будет $i$), и ``не правее'', чем $r$ для второго (индекс найденного элемента в массиве $b$ будет $j$). В ответ выводим $j-i+1$. На ответ затратится $\O(\log n)$
\end{solution}