%!TEX root = kotov.tex
\section{Task 1}
\begin{enumerate}[(a)]
    \item Введем $\widetilde{C} = \max\limits_{n}\left\{\frac{f(n)}{g(n)}\right\}$. При ограничениях, наложенных на ф-ции $f$ и $g$, $\widetilde{C} > 0$, тогда мы можем $\forall n$ с $\widetilde{C} > 0$ рассмотреть следующую цепочку: $\forall n \,\, f(n) \leq f(n) \leq \widetilde{C}g(n) \Longrightarrow f(n) \leq \widetilde{C}g(n)$. Таким образом, в определении О-большого можно пренебречь условием $\exists N:\ldots$ (это верно для обоих случаев: $\mathds{N} \to \mathds{N}$ и $\mathds{N}\to\mathds{R}_+$)
    \item Рассмотрим функции
    \begin{gather}
        f(n) = 
        \begin{cases}
            3, n = 1 \\
            n, n > 1    
        \end{cases} \\
        g(n) = n^2
    \end{gather}
    Очевидно, что в данном случае нельзя пренебречь условием $\exists N:\ldots$, так как вначале не для $\forall C f(n) < Cg(n)$. Если этим условием не пренебрегать, то $f(n) = o(g(n))$. Опять же, это верно для обоих случаев $\mathds{N} \to \mathds{N}$ и $\mathds{N}\to\mathds{R}_+$.
\end{enumerate}
P.S. Хотя формально, у нас не было ограничений на поведение функций $f$ и $g$, для случая $\mathds{N}\to\mathds{R}_+$, важно, чтобы они были достаточно хорошие (то есть, по видимому, не имели сингулярностей в конечных точках). Сделав поправку на ``природу'' исследуемых функций (оценка сложности алгоритма), мы можем вполне разумно считать, что с ростом $n$ функции также возрастают и не имеют каких-то странных разрывов по типу $\frac{1}{x - x_0}$