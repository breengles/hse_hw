%!TEX root = kotov.tex
\section{Task 5}
\begin{task}
    Про людей уважаемых.
\end{task}

\begin{solution}
    Давайте разделим наших людей на два класса: уважаемые (c $b_i \geq 0$), и менее уважаемые ($b_i < 0$). Обозначим людей уважаемых массивом $c_+$ и менее уважаемых как $c_-$. Идея в том, чтобы сначала попытаться набрать как можно больше людей уважаемых, а потом на все деньги (авторитет) набрать менее уважаемых.

    Отсортируем людей уважаемых по $a_i$. 
    \begin{upd}
        Люди у нас характеризуются кортежем из двух чисел $c_i = (a_i,\,b_I)$: требуемого авторитета $a_i$ и авторитета добавочного $b_i$. Уважаемых людей мы сортируем по возрастанию первого ключа.
    \end{upd}
    Будем набирать людей уважаемых, пока можем себе это позволить, в конце, если не дошли до конца массива $c_+$, то утверждаем, что всех нанять не сможем, так как всех уважаемых людей, которых могли нанять уже наняли, а менее уважаемые люди авторитет не поднимают.
    \begin{upd}
        Очевидно, что с самого начала набирать людей с отрицательным добавочным авторитетом странно, так как в таком случае мы точно уменьшаем свой текущий авторитет, а, набирая людей с неотрицательным авторитетом, мы точно не уменьшаем свой текущий авторитет.
        То есть, если бы мы могли набрать всех уважаемых людей (с неотрицательным добавочным авторитетом), то это было бы наибольшее значение суммарного авторитета в задаче.
        Теперь, в процессе набирания уважаемых людей мы можем попасть в ситуацию, когда на очередного (и всех последующих, так как они отсортированы по требуемому авторитету) уважаемого человека нашего авторитета не достаточно, а позволить набрать себе мы можем только менее уважаемых людей, то есть только понизить свой текущий авторитет, что явно не приближает нас к следующему уважаемому человеку.
        То есть Если мы не смогли набрать кого-то из уважаемых в процессе вербовки этих самых уважаемых людей, то мы достигли потолка нашего возможного авторитета (это могло вовсе произойти на старте вербовки, то есть мы вообще ни одного уважаемого человека завербовать не можем).
    \end{upd}

    \begin{upd}
        Осознал, что нанимать менее уважаемых людей сложнее, чем уважаемых.
        Давайте рассмотрим какую-то оптимальную последовательность вербовки, из нее поймем, что нам нужно делать: будем считать, что уважаемые люди все наняты, чтобы сфокусироваться только на менее уважаемых, количество которых обозначим за $m$.
        Пусть у нас есть некоторая оптимальная последовательность вербовки.
        Чтобы мы могли нанять вообще всех менее уважаемых людей (напоминаю, для них $b_i < 0\,\,\forall i\in[1,\ldots,m]$).
        Тогда на любом шаге вербовки должно быть выполнено
        \begin{gather}
            a_i \leq A + \sum_{j=1}^{i-1}b_i \\
            a_i + b_i \leq A + \sum_{j=1}^{i}b_i
        \end{gather}
        Последнее неравенство намекает, что можно попробовать будет сортироваться по $c_i = a_i + b_i$
        Пусть у нас есть последовательность $(1,2,\ldots,i-1,i,i+1,\ldots,m)$, причем $c_i < c_{i+1} \leftrightarrow a_i + b_i < a_{i+1} + b_{i+1}$.
        Для такой оптимальной вербовки мы имеем $a_i + b_i < a_{i+1}+b_{i+1} \leq A + \sum_{j=1}^{i+1}b_j$, отсюда мы можем получить, что $a_i \leq A + \sum_{j=1}^{i-1}b_j + b_{i+1}$.
        
        С другой стороны, мы хотели бы поставить $i$-ого и $(i+1)$ -ого человека в правильном друг относительно друга порядке ($c_i > c_{i+1}$): $(1,2,\ldots,i-1,i+1,i,\ldots,m)$.
        Из оптимального решения можно получить, что $a_{i+1} \leq A + \sum_{j=1}^{i}b_j < A + \sum_{j=1}^{i-1}b_j$ (последнее неравенство выполнено, так как $b_i < 0$).
        То есть в таком порядке мы до сих пор сможем нанять $i+1$ человека в отсортированном порядке, и при этом нам хватит авторитета, чтобы нанять $i$-ого человека (здесь как бы указаны старые индексы, но реально они идут в последовательности $\ldots,i+1,i,\ldots$), так как $a_i \leq A + \sum_{j=1}^{i-1}b_j + b_{i+1}$, где как раз явно указано изменение нашего авторитета после вербовки предыдущего человека.
    \end{upd}
    Теперь мы можем отсортировать менее уважаемых людей в порядке убывания $c_i = b_i + a_i$, и вербовать людей в таком порядке.
    
    Тогда в конце, если дошли до конца массива $c_-$, заявляем, что мы единственная банда в городе, если нет, то мы просто самая неавторитетная банда в городе.

    В процессе мы за линию разбивали один раз исходный массив уважаемых (или не очень) людей, и два раза друг за другом сортировали за $\O(n\log n)$, будет $\O(n\log n)$.
    
    \begin{remark}    
        Заодно бесплатно получили подпункт b), а именно, мы в этой же процедуре, пока собираем людей уважаемых увеличиваем счетчик завербованных людей, если не можем очередного человека уважаемого нанять (или они закончились), то приступаем к вербовке людей менее уважаемых, каждый раз увеличивая счетчик.
        Этот счетчик и будет говорить нам, сколько мы можем набрать людей.
    \end{remark}
\end{solution}