%!TEX root = kotov.tex
\section{Task 1}
\begin{task}
    Про разбиение на циклы.
\end{task}

\begin{solution}
    Сделаем граф из двух долей, в каждую поместим все вершины исходного графа. Соединим доли между друг другом ребрами единичной пропускной способности так, как они были соединены в исходном графе, но с условием, что ребра выходят только из левой доли.
    
    Найдем максимальный поток в таком графе.
    Если он равен $|V| = n$, то мы смогли разбить на граф на циклы, если нет, то невозможно разбить граф на циклы.

    Почему это работает? Рассмотрим конечный корректный ответ.
    Первое наблюдение: если пустить поток в цикле из истока в исток (т.е. как бы мы вышли их вершинки и в нее же вернулись), то величина потока через каждую вершину цикла одна и та же и равна $1$, если пропускные способности $1$.
    Второе наблюдение: в цикле если поток дошел, например, из вершины $1$ до вершины $2$, а из нее дошел до вершины $3$, из которой вернулся в вершину $1$, то на нашем рассматриваемом графе это бы значило, что у нас есть как бы цепочка: рассмотрим ребро $v_i \rightarrow v_j$, посмотрим, куда идет поток из $v_j$ в левой доли, пусть идет в $v_k$, опять посмотрим куда идет ребро из $v_k$ в левой доли и т. д., для корректного цикла мы в конце обязательно вернемся в $1$-ую вершину.
    
    Теперь, когда мы одновременно из вершины сливаем приходящий поток, и пускаем новый поток, то это эквивалентно предыдущей схеме, разве что мы как бы можем посчитать количество вершин в цикле, соответственно, если суммарное количество вершин в циклах равно полному количеству вершин в графе (можно опять же посмотреть на корректное разбиение и понятно, что это должно быть), то разбить можно. Восстановить непосредственно само разбиение можно по процедуре рассматривания куда идет поток от вершины из левой доли, если мы в нее пришли в правой доли (``цепочка'' описанная ранее).

    \begin{upd}
        Забыл про оценку сложности: для построенной сети ФФ отработает за $\O(|f|E')$, где $|f| = V$, так как мы ожидаем именно такой поток, а $E'$ складывается из количества ребер в исходном графе $E$, количества ребер, ведущие из исток в левую долю (их $V$ штук), и ребер, ведущие из правой доли в сток (их тоже $V$ штук), тогда сложность можно оценить как $\O(|f|E') = \O\big(V(2V + E)\big)$.
        
        Теперь важное уточнение, исходно я считал, что наш граф ну хотя бы связный, тогда $E \geq V - 1$ (для разных компонент тоже можно очень грубо оценить чем-то типо $E \geq \frac{V}{2}$), хотя для циклов мы можем гарантировать $E \geq V$, тогда в полученной оценке сложности доминирующим окажется вклад $VE$, следовательно итоговая сложность $\O(VE)$.
    \end{upd}
\end{solution}