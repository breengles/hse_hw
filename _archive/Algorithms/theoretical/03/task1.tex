%!TEX root = kotov.tex
\section{Task 1}
\begin{task}
    Оцените время работы детерминированного алгоритма поиска порядковой статистики, если вместо пятерок разбивать элементы на
    \begin{enumerate}[(a)]
        \item семерки.
        \item тройки.
    \end{enumerate}
\end{task}

\begin{solution}
    \begin{enumerate}[(a)]
        \item 
        Рассмотрим для начала некий общий случай разбиения на $k$ штук. Делаем такую же процедуру, как на паре.
        Может оценить количество элементов в левом блоке как $\geq \frac{k+1}{2}\frac{n}{2k}=\frac{n(k+1)}{4k}$.
        Тогда в наибольшей оставшейся части можно оценить как $\leq n - \frac{n(k+1)}{4k} = \frac{n(3k-1)}{4k}$
        \begin{remark}
            Проверим, что это сходится с полученными оценками для $k=5$: smaller block $\geq \frac{3n}{10}$, greater block $\leq \frac{7n}{10}$
        \end{remark}
        Теперь посмотрим такую же оценку для $T(n) \leq Cn + T\left(\frac{n}{k}\right) + T\left(n\frac{3k-1}{4k}\right)$. Попробуем оценить $T(n) \leq AC'n$, где $C'>C$.
        Так же по индукции будет проверять: рассмотрим переход
        \begin{gather}
            T(n) \leq Cn + T\left(\frac{n}{k}\right) + T\left(n\frac{3k-1}{4k}\right) \leq Cn + \frac{n}{k}AC' + \frac{3k-1}{4k}nAC' = \\
            = \left(C + \frac{AC'}{k} + \frac{3k-1}{4k}AC'\right)n \leq \left(1 + \frac{A}{k} + \frac{3k-1}{4k}A\right)C'n = \\
            = \frac{4k + 4A + (3k - 1)A}{4k}C'n = \frac{3kA + 3A + 4k}{4k}C'n
        \end{gather}
        Потребуем, чтобы $\frac{3kA + 3A + 4k}{4k} = A$, тогда $A = \frac{4k}{k-3}$.
        Таким образом, имеем, что для такой выбранной $A$ оценка $T(n) \leq \frac{4k}{k-3}C'n$ верна, то есть $T(n) = \O(n)$
        \begin{remark}
            Проверим, совпадет ли эта константа с док-вом для $k=5$: $A=10$.
        \end{remark}
        Эта оценка подойдет для $k > 3$.
        \item
        Рассмотрим теперь отдельно случай $k=3$. Во-первых, из предыдущего можно заметить, что для этого случая оценка $T(n) \leq Bn$ не работает. То есть, можно утверждать, что $T(n) \neq \O(n)$. Также, очевидно, что можно решить задачу за $\O(n\log n)$ (отсортировать исходный массив).

        Более того, аналогичными выкладками из предыдущего пункта можно показать, что $T(n) = \Omega(n\log n)$. Комбинируя эти два результата, мы может утверждать, что $T_{k=3}(n) = \Theta(n\log n)$
    \end{enumerate}
\end{solution}