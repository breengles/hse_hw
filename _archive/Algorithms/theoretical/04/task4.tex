%!TEX root = kotov.tex
\section{Task 4}
\begin{task}
    Дан массив из $n$ целых чисел, число $d$ и число $k$. Найти подпоследовательность длины $k$ с максимальной суммой элементов при условии, что соседние элементы в ней должны отличаться не более чем на $d$. $\O(n^2k)$.
\end{task}

\begin{solution}
    Заведем массив $d_{ij}$ --- максимальная сумма такой последовательности $1<j<k$ штук элементов от $0$-ого до $i$-ого исходного массива, что она заканчивается $a_i$-ым элементом исходного массива.
    Естественным ограничением будет $d_{i1} = a_i$ $\forall i \in [1, n]$, так как только сам элемент и является такой последовательностью длины $1$.
    Теперь надо поступить почти также как с поиском возрастающей последовательности:
    $d_{ij} = \max\limits_{l<i}(d_{ij}, d_{l,j-1} + a_i)$ $\forall j \in [2, k]$, то есть если существует такая последовательность длины $j-1$, оканчивающаяся раньше $i$, что сумма ее значения и текущего нового элемента больше, чем текущее максимальное значение для последовательности, заканчивающейся на текущем элементе, то стоит обновить значение для текущего элемента. Так как тут добавилась еще и пробежка по всем $j \in [2, k]$, то сложность будет уже $\O(n^2k)$.
    \sout{В ответ надо будет выдать $\max\limits_i(d_{ik})$}
    \begin{upd}
        Видимо, можно поступить как и в предыдущей задачи: надо будет сохранять указатели на предыдущие элементы последовательности. Вообще, казалось бы, надо для каждого $k$ это сделать, но мы все равно интересуемся каким-то конкретным $k$, так что все должно быть ок даже по памяти (то есть это может быть один массив, который с ростом $k$ тоже растет). То есть опять таки, найдя $\max\limits_i(d_{ik})$ по указателям на предыдущий элемент восстановим последовательность.
    \end{upd}
\end{solution}