%!TEX root = kotov.tex
\section{Task 6}
\begin{task}
    Клетки поля $n \times 5$ покрашены в чёрный и белый цвета. Будем называть получившийся узор красивым, если он не содержит одноцветного квадрата $2 \times 2$. Вычислите число красивых узоров по модулю небольшого простого числа за время $\O(\log n)$.
\end{task}

\begin{solution}
    Попробуем свести это дело к какому-то рекуррентному соотношению.
    Рассмотрим столбцы длины 5 (то есть наша плитка будет расти слева направо). Всего существует 32 раскраски таких столбцов. Заведем массив $a_{ik}$ --- количество раскрасок $k$ столбцов, заканчивающихся каким-то конкретным $i$-ым столбцом. Также заведем еще массив $d_{ij}$ --- говорит нам, можем ли мы поставить раскрашенный столбец $i$ рядом с раскрашенным столбцом $j$ (0, если не можем, 1, если можем), чтобы узор оставался красивым. Естественное ограничение $a_{i1} = 1$, так как существует ровно одна какая-то конкретная раскраска одного столбца, всегда же $1 \leq i \leq 32$.
    
    Теперь зададимся вопросом, как связано количество раскрасок $a_{ik}$ с предыдущими раскрасками, то есть какова динамика?
    Рассмотрим $a_{ik} = \sum\limits_{j=1}^{32}a_{j,k-1}d_{ji}$, то есть мы перебираем всевозможные раскраски последнего, $j$-ого столбца и проверяем, можем ли мы поставить текущий новый $i$-ый столбец в такую комбинацию, ну и естественно просуммировать подходящие варианты красивых раскрасок.

    Теперь отнесемся к $i$ у $a$ как к индексу некоторого вектора, тогда динамику можно переписать так $\vec{a}_k = \mathcal{D}\vec{a}_{k-1}$, где матрица $\mathcal{D}$ --- матрица $d$, причем симметричная.
    То есть мы получили хороший линейный рекуррент в матричной форме.
    
    Определим начальные условия, то есть $\vec{a}_1$. Как раньше уже упоминалось, у нас есть ограничение $a_{i1} = 1$, тогда $\vec{a}_1 = (1,\ldots,1)^T$, теперь мы можем окончательно сформулировать нашу задачу:
    $\vec{a}_n = \mathcal{D}^n\vec{a}_1$. Возводить матрицу в степень мы умеем по следующей схеме: $\mathcal{D}^n = \mathcal{D}^{n/2}\mathcal{D}^{n/2} = \mathcal{D}^{n/4}\mathcal{D}^{n/4}\mathcal{D}^{n/4}\mathcal{D}^{n/4}$, то есть за $\O(\log n)$.
\end{solution}