%!TEX root = kotov.tex
\section{Task 2}
\begin{task}
    Дан набор нечестных монеток с вероятностью выпадения орла $p1, p2, \ldots, pn$. Требуется посчитать вероятность выпадения ровно $k$ орлов за $\O(n \cdot k)$. Операции над числами считать выполнимыми за $\O(1)$.
\end{task}

\begin{solution}
    Рассмотрим следующий массив $d_{kn}$, означающий вероятность выбросить $k$ орлов на $n$-ом броске.
    Например, $d_{01} = 1 - p_1$, $d_{12} = p_1(1-p_2) + p_2(1-p_1)$
    Можем задаться вопросом, как эта вероятность связана с предыдущими бросками: либо на $n-1$ броске уже было выброшено $k$ орлов, и тогда нам надо не выбросить орла не $n$-ом броске, либо на $n-1$ броске было выброшено $k-1$ орел (если меньше, то заведомо на последнем броске невозможно получить достаточно количество орлов), тогда нам надо выбросить еще одного орла, то есть динамика такая:
    \begin{equation}
        d_{kn} = (1 - p_n)d_{k,n-1} + p_nd_{k-1,n-1}
    \end{equation}
    то есть мы пробегаемся по всем $n$ $k$ раз (ну что-то порядка этого числа раз, то есть $\O(nk)$ раз), делая какие-то константные действия, итоговая сложность будет $\O(nk)$, в ответ уйдет значение $d_{kn}$
    
    \begin{upd}
        Граничные условия можно поставить следующие:
        $d_{00} = 1$, $d_{kk} = \prod\limits_{i=1}^kp_i$ $\forall k\in[1,n]$, еще можно взять кусочки из примера.
    \end{upd}
\end{solution}