%!TEX root = kotov.tex
\section{Task 1}
\begin{task}
    \textit{Есть $m$ стойл с координатами $x_1, \ldots, x_m$ и $n$ коров. Расставить коров по стойлам
	(не более одной в стойло) так, чтобы минимальное расстояние между коровами было максимально.
	$\O(m (\log{m} + \log{x_{\max}}))$.}
\end{task}
\begin{solution}
Отсортируем массив координат стойл, например, MergeSort'ом за $\O(m\log m)$.
Зададимся какой-то $\Delta$ -- минимальное расстояние между коровами.
Очевидно, что если мы зададимся очень маленьким минимальным расстоянием, то можно уместить всех коров, а с увеличением этого расстояния в какой-то момент мы не сможем расставить всех коров, чтобы минимальное расстояние между ними было равно заданному $\Delta$.

Проверить, что коров расставить можно, можно за линейное время (банальный проход вдоль всего массива координат с расстановкой очередной коровы, если расстояние от предыдущего стойла, в которое корова была поставлена, не меньше $\Delta$), то есть за $\O(m)$, что уже съедается изначальной сортировкой. Также понятно, что если мы зададимся $\Delta = x_{\text{max}}$, то мы точно не сможем расставить так коров.
\begin{remark}
    Почему можно проверять возможность расстановки коров за линейное время?
    Рассмотрим следующую процедуру:
    \begin{enumerate}[1)]
        \item Заведем переменную \texttt{cows}, считающую количество нераспределенных коров и изначально равной \texttt{m}
        \item В уже отсортированном массиве $\{x\}^{n}_{i=1}$ будем идти слева направо и ``ставить'' коров в стойла (в первое стойло всегда ставим корову), если разность расстояния между текущим стойлом и стойлом, в которое в предыдущий раз была поставлена корова, не меньше заданной $\Delta$
        \item После каждой поставленной коровы будем уменьшать счетчик оставшихся коров \texttt{cows} на единицу
        \item Останавливаемся если дошли до конца массива $\{x\}^{n}_{i=1}$ или $\texttt{cows} == 0$
        \item Если в итоге $\texttt{cows} == 0$, то коров расставить можно, если же $\texttt{cows} != 0$, то коров расставить нельзя
    \end{enumerate}
    Данная процедура занимает не более, чем один пробег по массиву стойл, выполняемые операции сравнения, уменьшение счетчика выполняются за константное время, следовательно, за линейное время можем проверить возможность расставить коров в стойла.
\end{remark}
Представим это графически:
\begin{figure}[H]
    \centering
    \includegraphics[width=\textwidth]{pics/35a3cc4456c54a0a907b86962bda6512.pdf}
    \caption{График максимально возможного количества $A$ расставленных коров при заданном минимальном расстоянии $\Delta$.}
\end{figure}
Теперь запустим бинарный поиск с тем условием, чтобы искомая $\Delta$ была крайней правой из возможных на плато, это работает за $\O(m\log x_{\text{max}})$. Таким образом, данное решение работает за $\O(m\log m + m\log x_{\text{max}})$
\end{solution}