%!TEX root = kotov.tex
\section{Task 3}
\begin{task}
    \textit{Даны два массива $a$ и $b$ длины $n$, сгенерировать все
    попарные суммы $a_i + b_j$ в сортированном порядке.
    \begin{enumerate}
        \item За $\O(n^2 \log n)$.
        \item За $\O(n^3)$ с использованием $\O(n)$ дополнительной памяти.
        \item За $\O(n^2 \log n)$ с использованием $\O(n)$ дополнительной памяти.
        \item За $\O(n^3)$ с использованием $\O(1)$ дополнительной памяти.
    \end{enumerate}
    Здесь считайте, что дополнительная память --- количество чисел длины $\O(\log n)$, которые вы можете сохранить.}
\end{task}

\begin{solution}
\begin{enumerate}[a)]
    \item Составим сначала просто массив всевозможных парных сумм исходных двух массивов: ${a_i+b_j}$, эта операция делается за $\O(n^2)$. Теперь отсортируем этот массив, например, MergeSort'ом за $\O(n^2\log n^2) = \O(n^2\log n)$. Итоговая сложность будет мажорироваться именно второй операцией, то есть $\O(n^2\log n)$.

    \item Сортируем массив $a$ за $\O(n\log n)$. Затем для каждого $i=1,\ldots,n$ составляем массив $\{c_i\} = [a] + b_i$, это делается за $\O(n)$.
    
    Затем заводим дополнительный массив длинны $n$, в который будем складировать самые левые элементы всех $c_i$. За линейное время находим минимум в данном массиве, затем помещаем найденное число в результирующий массив, а из соответствующего $c_i$ этот элемент удаляем. Так надо сделать до тех пор, пока не иссякнут $c_i$, то есть порядка $n^2$ раз.

    Таким образом, мы вылавливаем всегда следующий минимальный элемент, который будет положен в результирующий массив. Итоговая сложность --- $\O(n^3)$, дополнительно был использован массив длины $n$ для хранения ``головы'' каждого $c_i$.

    \item В целом, делаем то же самое, что и в предыдущем пункте, разве что для нахождения минимума ``голов'' заведем min-кучу, в которой будем делать \texttt{extract\_min()} за $\O(\log n)$ (всего надо будет проделать $n^2$ раз, следовательно, $\O(n^2\log n)$), а после добавлять новый элемент из соответствующего $c_i$.
    
    \item (тут надо как-то придумать, как отловить неучтенный элементы за $\O(n)$ внутри цикла, может быть, но я пока над ней думаю)
    \begin{enumerate}[1.]
        \item Отсортируем оба изначальных массива
        \item положим $\texttt{current\_min}=a_1 + b_1$
        \item положим $i=j=1$
        \item то есть
        \begin{lstlisting}
current_min = a[1] + b[1]
i = 1
j = 1
while i != n and j != n:
    if a[i] + b[i+1] <= a[i+1] + b[j]:
        current_min = a[i] + b[j+1]
        j += 1
    else:
        current_min = a[i+1] + b[j]
        i += 1
        \end{lstlisting}
        \item \texttt{current\_min} сгружаем в результирующий массив до цикла и на каждой итерации.
        \begin{remark}
            Цикл отработает за $\O(n^2)$, дополнительно, формально, использована лишь одна переменная. Проблема в том, что там могут потеряться некоторые элементы: их надо как-то отслеживать. Судя по всему, надо как-то пробежаться по всем ним за линейное время.
        \end{remark}
    \end{enumerate}
\end{enumerate}
\end{solution}