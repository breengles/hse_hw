%!TEX root = kotov.tex
\section{Task 2}
\begin{task}
    Даны $n$ гномов. Если $i$-го гнома укладывать спать $a_i$ минут, он потом спит $b_i$ минут. Можно ли сделать так, чтобы в какой-то момент все гномы спали? $\O(n\log n)$.
\end{task}

\begin{solution}
    Заведем массив $c_i = a_i + b_i$ и отсортируем этот массив по убыванию. Набираем до тех пор, пока либо не дошли до конца массива, либо время сна очередного гнома меньше, чем время укладывания следующего. На эту процедуру потратим $\O(n\log n + n) = \O(n\log n)$.
    \begin{upd}
        В конце этого цикл надо будет проверить, дошли ли мы до конца массива, то есть если мы вылетели из-за условия, что время сна очередного гнома меньше, чем время укладывания следующего, то нельзя уложить гномов так, чтобы была свободна минутка.
    \end{upd}

    Корректность: пусть есть некоторое оптимальное решение, и при этом $b_k + a_k < b_{k+1} + a_{k+1}$. Тогда в таком решении $b_k > \sum\limits_{i=k+1}^n a_i$ и $b_{k+1} > \sum\limits_{i=k+2}^n a_i$. Прибавим, соответственно, $a_k$ и $a_{k+1}$: $b_k + a_k > \sum\limits_{i=k}^n a_i$ и $b_{k+1} + a_{k+1} > \sum\limits_{i=k+1}^n a_i$. Тогда, использую предположение о положении гномов в неотсортированном массиве: $b_{k+1} + a_{k+1} > b_k + a_k > \sum\limits_{i=k}^n a_i > \sum\limits_{i=k+1}^n a_i$, потому что иначе гному не уложены так, что есть свободное время. Теперь поменяем местами этих двух гномов, это все равно что они поменяются местами в левой части цепочки неравенств, то есть и для поменянных местами (отсортированных по убыванию) гномов это тоже выполняется.
    \begin{upd}
        То есть имеется в виду, что пусть у нас есть какая-то оптимальная (но неотсортированная по убыванию) последовательность гномов $(1,2,\ldots,i,i+1,\ldots,n)$, для нее выполнены неравенства, указанные выше. Теперь переставим $i$-ого и $(i+1)$-ого гномов местами: $(1,2,\ldots,i+1,i,\ldots,n)$. Для нее гномы $i$ и $i+1$ стоят в правильном относительно друг друга порядке по убыванию, и из соотношений для исходной оптимальной расстановки можно получить следующие неравенства $b_{i+1}+a_{i+1} > \sum\limits_{j=i}^{n}a_j$, то есть $i$-ый гном влезет в $i+1$-ого, а так как в оптимальной расстановки в $i$-ого гнома влезали все гномы с большими номерами, то явно влезают все гномы и без $i+1$-ого, то есть, расставив этих гномов в правильном порядке, для гномов с большими номерами мы условие не нарушили. Стоит отметить, что для гномов с меньшими номерами ничего вовсе не изменилось от такой перестановки.
    \end{upd}

    В итоге, если мы смогли в отсортированном массиве дойти до конца, то смогли уложить, и надо их укладывать в порядке отсортированного массива $c$
    \begin{remark}
        Естественно, что надо будет еще сохранить исходные номера гномов.
    \end{remark}
\end{solution}