%!TEX root = kotov.pdf
\section{Task 4}
\begin{task}
    Про антиклики,
\end{task}

\begin{solution}
    \begin{enumerate}[a)]
        \item Этот пункт будет сделан в стиле древнего способа деления числе: то есть что такое целая часть $a/b$? Это то, сколько раз (минус один) мы можем вычесть из $a$ число $b$ пока результат не станет $\leq 0$.
        Поступим в этой задаче самым наивным образом, а именно возьмем первую вершину, ``удалим'' саму вершину (но возьмем ее в антиклику) и всех ее соседей из графа, то есть из условия задачи мы удалим не более чем $d+1$ вершин, перейдем к следующей вершине и поступим таким же образом, и т.д. В итоге мы сделаем не менее $\frac{n}{d+1}$ действие, на котором будем брать по одной вершине, то есть в итоге в нашей антиклике будет содержаться как минимум $\frac{n}{d+1}$ вершин (то есть это старинный способ деления, только нашем случае нам можно не вычитать единицу из ``целой части'', так как хотя бы одну вершину антиклика все-таки содержит).
        
        По сложности мы должны будем пройтись один разочек по исходному массиву, для каждой вершины делая в худшем случае $d+1$ действие, связанное, например, с тем, что мы запоминаем какие вершины уже удалены (что-то типа массива $0$ и $1$ для исходного массива, где изначально все элементы $1$, а $0$ означает, что мы удалили вершину из множества). В итоге сложность будет $\O((d+1)\frac{n}{d+1}) = \O(n)$ в худшем случае.
        \item *грустный смайлик.png*
    \end{enumerate}
\end{solution}