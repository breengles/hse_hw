%!TEX root = kotov.tex
\section{Task 3: Т9}
\begin{solution}
    Заводим бор, при этом в терминалах храним еще дополнительную информацию --- частоту, и еще из них будут ссылки на 5 самых частых.
    Посмотрим на первый запрос: спускаемся по бору, находим слово, увеличиваем его частоту.
    Для каждой вершины на обратном пути (даже не терминальных!) к корню после этого надо проверить, надо ли обновить ссылки на самых частых. Это можно сделать через сравнение минимальной частоты среди текущих 5, и если надо, то обновляем ссылку на только что ``увеличенную'' вершину, каждая такая операция стоит нам $\O(1)$, кажется, что это дорого, но у нас даже суммарная длина слов не больше $10^6$.

    \begin{remark}
        Глубже спускаться смысла нет, так как они явно не суффиксы слова, у которого мы увеличили частоту.
    \end{remark}

    Теперь, когда приходит второй запрос, мы спускаем по бору, находим вершину, в которой надо остановиться (это может быть не терминальной вершиной), смотрим на 5 самых встречаемых (у нас есть вот такие вспомогательные ссылки).
    Чтобы вывести эти 5 мы переходим по этим ссылочкам, которые указывают в терминальные вершины, поднимаемся от них к корню, это будет как бы слово наоборот (в смысле обычное слово получается у нас спуском из корня в терминал, а если мы поднимаемся из терминала к копню, то мы получаем слово, записанное наоборот), следовательно, надо просто развернуть прочитанное и выдать пользователю. Такое, вообще говоря, будет нам стоить О-большое от суммарной длины этих самых 5 частых слов.

    \begin{remark}
        Ясно, что быстрее нельзя, так как хотя бы сам вывод будет столько стоить.
    \end{remark}
\end{solution}