%!TEX root = kotov.tex
\section{Task 4: Семейства функций}
\begin{solution}
    Важное предположение, я считаю в дальнейшем, что функции переводят каждый элемент их левого множества, в какие-то (необязательно различные, но единственные) элементы правого множества, вроде бы, это называется инъекцией.
    \begin{enumerate}
        \item Представим это себе как две доли графа, слева у нас $2^n$ вершин, справа --- $2^m$. Тогда нам комбинаторно надо посчитать кол-во различных ребер из левой доли в правую, т.е. по отношению каждой вершине справа задать вопрос, кто в нее приходит, тогда кол-во инъективных отображению $2^{m2^n}$.
        \item По определению, посчитаем вероятность, что $\forall x_1 \neq x_2: h(x_1) = h(x_2)$. Такая вероятность равна $P = \frac{\#\{h(x_1) = h(x_2)\}}{|H|} = 2^{m(2^n - 1)}/2^{m2^n} = \frac{1}{2^m} = \frac{1}{|\mathbb{F}_2^m|}$. То есть по определению для любых $n$ и $m$. Так же показывается и 2-независимость $\frac{\#\{h(x_1) = y_1, h(x_2) = y_2\}}{|H|} = \frac{2^{m(2^n - 2)}}{2^{m2^n}} = \frac{1}{(2^m)^2} = \frac{1}{|\mathbb{F}_2^m|^2}$
        \item Хм, стоит отметить, что бесконечности бывают разные, остановимся на случае счетного множества. Тогда, кажется, что сломается условие на произвольные $x$, но что-то не получается формально показать.
    \end{enumerate}
\end{solution}

