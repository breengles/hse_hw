%!TEX root = kotov.tex
\section{Task 1}
\begin{task}
    Найти в неорграфе простой цикл через данную вершину за $\O(E)$.
\end{task}

\begin{solution}
    Запустим из данной вершины \texttt{dfs}, тогда мы либо найдем цикл, либо пройдемся по всем ребрам, при мы не запускаемся от вершин, которые не достигли из данной стартовой вершины.
    Почему мы можем позволить себе именно такой запуск?
    Потому что поиском из вершины получаем все достижимые вершины из данной вершины, то есть если вершина содержится в простом цикле, т.е. есть путь из нее в нее же по уникальным единожды посещенным (кроме нее самой) вершинам, то запуск поиска из другой вершины, принадлежащей этому же пути, приводит к тому же ответу о принадлежности заданной вершины циклу.

    Теперь, если мы запустили поиск из вершины, не принадлежащей циклу, содержащему исходную вершину, но достижимому (в смысле цикл достижим) из стартовой вершины, то по достижении этого цикла мы найдем исходную вершину в нем, если она там есть. Так можно проделать для всех циклов, соответственно, если вершина не принадлежит циклу, то мы бы и так ее не нашли.
    Случаи, когда мы бы не достигали исходной вершины при запуске из какой-то другой вершины, не очень интересные, так как если мы не достигли её просто поиском, то уже и в циклах, которые мы попутно бы нашли, исходная вершина точно не содержится.
    Следовательно, почему бы сразу не запустить поиск из исходной вершины.
\end{solution}

\begin{upd}
    Попробую разбить утверждения для чистоты:
    \begin{enumerate}
        \item Если вершина содержится в цикле, то она содержится в компоненте связности, содержащей этот цикл.
        \item Чтобы определить, содержится вершина в цикле, нам, естественно, нет смысла запускать поиск в глубину в других компонентах связности.
        \item Ну и раз уже мы из любой вершины (в неорграфе) с поиском в глубину достигаем все вершины той же самой компоненты связности, то можем позволить себе просто запускаться сразу от заданной в условии вершины.
        \item В плане пройдемся по всем ребрам имеется в виду, что мы посмотрим в худшем случае все ребра компоненты связности, к которой принадлежит стартовая вершина.
    \end{enumerate}
\end{upd}