%!TEX root = kotov.tex
\section{Task 1.1 (Случайный доп.)}
\begin{task}
    У каждой вершины не более $3$ врагов. Вражда – симметричное отношение. Разбить вершины на $2$ доли так, чтобы с вершиной в долю попало не более $1$ врага. $\O(V + E)$.
\end{task}

\begin{solution}
    Честно, я случайно начал ее решать, но раз уже есть какое-то решение, то приведу его.

    Представим эти отношения враждебности как ребра на графе, т.е. из условия следует, что степень каждой вершины не больше $3$.
    Заведем два цвета для двух долей: красный и синий.
    И будем отслеживать активный цвет, например, на старте красный.
    Запустим поиск в глубину и будем красить вершины в активный цвет. Если наткнемся на ситуацию, в которой число смежных вершин, покрашенных в активный цвет больше $1$, то меняем активный цвет на другой.

    Почему это, вероятно, работает?
    Рассматриваем только связные графы, иначе можно было бы разбить задачу на несколько подзадач на каждой из компонент связности. Поиском в глубину можно построить своего рода подвешенное дерево на вершине, из которой запускаемся. Если такое дерево содержало бы вообще все ребра исходного графа, то было бы все просто, так как мы бы просто красили ветки в разные (чередующиеся) цвета (слова передаются с трудом). Но такое дерево не видит ребер, которые могут идти в одной ветви из одной вершины в другую (на семинаре уже ``показали'', что ребер между ветками быть не может). Тогда если бы у нас была ветка одного цвета, то мы бы покрасили ее неправильно, ну как раз для этого мы в какой-то момент меняли цвет раскраски, чтобы такие вершины были разного цвета. Так как у нас максимум степень вершины $3$, то у нас как раз хватит двух цветов, чтобы покрасить ее и ее соседей так, чтобы не было больше одной одноцветной пары для одной и той же вершины (опять сложное перекладывание мысли в слова).
\end{solution}