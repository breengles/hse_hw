%!TEX root = kotov.tex
\section{Task 3}
\begin{task}
    Дан массив из $n$ целых чисел и число $d$. Найти подпоследовательность максимальной длины с условием, что соседние элементы в ней должны отличаться не более чем на $d$ за $\O(n^2)$.
\end{task}

\begin{solution}
    Поступим как с поиском возрастающей последовательности, но с другим условием (в поиске возрастающей последовательности мы смотрели, что $a_i < a_j$), а именно $\left|a_j - a_i\right| \leq d$.
    То есть составим массив $d_i$ --- длина самой длинной последовательности, разница между соседними элементами которой не больше $d$, при этом эта последовательность заканчивается на $a_i$-ом элементе исходного массива.
    Динамика: $d_i = 1 + \max\limits_{1<j<i}(d_j)$ при условии, что $\left|a_j - a_i\right| \leq d$.
    Если такого $j$ нет, то $d_i = 1$. Это делается за $\O(n^2)$.
    \sout{В ответ выводим $\max\limits_{i}d_i$, что делается еще за $\O(n)$ но это уже не важно.}
    \begin{upd}
        Значит, нам надо будет запоминать последовательность, например, запоминать для каждого $i$ из какого элемента $j$ мы пришли (пусть это будет массив \texttt{previous[i]}), тогда, найдя максимум $d_i$, с помощью таких указателей на предыдущий элемент мы сможем восстановить саму последовательность.
    \end{upd}
\end{solution}