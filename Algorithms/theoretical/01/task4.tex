%!TEX root = kotov.tex
\section{Task 4}
\begin{center}
$[a_1, \ldots, a_n] \in \mathds{N}, S \in \mathds{N}$
\end{center}
Введем $r_i = \max\limits_{r}\left\{\sum\limits_{j=i}^{r}a_j \leq S\right\}$, то есть это максимальное количество элементов справа от $i$-ого, которое можно взять в сумму и не ``перепрыгнуть'' через $S$.

Начнем находить такие $r_i$:
\begin{enumerate}
    \item Поставим два указателя (ал-я левый и правый) на первый элемент
    \item Пока $\sum\limits_{i=1}^{r}a_i \leq S$ будем сдвигать правый указать ($r +\!\!= 1$) до тех пор, пока не дойдем до некоторого элемента, добавление в сумму которого превзойдет $S$
    \item Откатим правый указатель на 1 влево: индекс элемента, на который указывает правый указатель и есть искомый $r_1$ (в этот момент мы могли уже получить искомую сумма, поэтому проверим, $\sum\limits_{i=1}^{r_1}a_i = S?$)
    \item Если нет, то подвинем левый указатель (он все это время оставался на первом элементе) на единицу вправо ($l +\!\!= 1$).
    \item Так как все элементы --- натуральные числа, то $\sum\limits_{i=2}^{r_1}a_i < \sum\limits_{i=1}^{r_1}a_i$
    \item Проверим, можем ли подвинуть правый указатель, если да, то продолжаем так двигать, пока сумма не окажется большей $S$, затем повторим сдвижку левого указателя. Если изначально нельзя было сдвинуть правый указатель, то $r_2 = r_1$ и мы подвинем левый указатель еще раз на $1$ направо и т.д.
    \item В итоге, рано или поздно, либо на каком-то элементе получим сумму, равную $S$, либо упремся правым указателем в конец массива (тут мы можем констатировать факт что, либо последняя найденная сумма равна $S$, либо такого искомого отрезка для данного $S$ в массиве нет), либо два указателя встретятся на одном элементе (что значит, что этот элемент сам по себе больше, чем $S$). В последнем случае, сдвинем оба указателя на следующий элемент и будем продолжать подобную схему далее.
\end{enumerate}
В описанной схеме, в худшем случае мы сделаем два пробега по всему массиву (случай, когда у нас только последний элемент равен $S$, а остальных не хватает, чтобы набрать необходимую сумма).