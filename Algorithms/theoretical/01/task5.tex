%!TEX root = kotov.tex
\newpage
\section{Task 5}
\begin{enumerate}[(a)]
    \item Используем стэк $s$:
    \begin{itemize}
        \item Для каждого элемента $a_i$ исходного массива делаем следующие (проходим массив слева напарво, то есть $i = 1 \ldots n$):
        \begin{enumerate}[1)]
            \item Пока $s$ непуст и $s_{\text{top}} \geq a_i$: S.pop
            \item Если $s$ пуст, то у $a_i$ нет элементов, которые левее и меньше его
            \item Иначе, самый правый элемент, который левее и меньше $a_i$ есть $s_{\text{top}}$
            \item s.push($a_i$)
        \end{enumerate}
    \end{itemize}
    \item Аналогично предыдущему пункту, только теперь массив проходим справа налево, то есть $i = n \ldots 1$
    \item С использованием двух предыдущих пунктов, мы можем найти для каждого элемента $a_i$ соответствующие $l_i$ и $r_i$, в интервале которых $a_i$ будет минимальным. Составим $\pi_i = (r_i - l_i + 1)\cdot\min\limits_{j\in[l_i,r_i]}a_j$ и сохраним их вместе с парой $(l_i, r_i)$, то есть что-то типа кортежа $\Pi_i = (\pi_i,\, l_i,\, r_i)$. Найдем $\max\limits_{i}\Pi_i[0]$. Соответствующие $l_i$ и $r_i$ и будут искомыми.
    \item Аналогично предыдущему, только если в прошлой задаче ``весом'' каждого элемента была длина промежутка, на котором он минимален, то теперь вес --- сумма всех элементов на этом промежутке.
\end{enumerate}

