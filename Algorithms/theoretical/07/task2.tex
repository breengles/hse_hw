%!TEX root = kotov.tex
\section{Task 2: Расстояния в меняющемся графе}
\begin{task}
	Нужно научиться на запрос <<уменьшился вес ребра>> за $\O(V^2)$ пересчитывать матрицу расстояний. 
	Считайте, что в графе не было и не появилось отрицательных циклов.
\end{task}

\begin{solution}
    Рассмотрим пути, который проходили через ребро $e = (cd)$ до изменения веса ребра (считаем, что матрица расстояний у нас уже была насчитана). Для таких путей верно, что $d[a][b] = d[a][c] + w_e + d[d][b]$.
    
    Теперь пусть изменился вес ребра $e$ (даже если оно нам в запросе не дается, то за $\O(E) \subseteq \O(V^2)$, мы его найдем). Рассмотрим всевозможные (переберем) вершины $a$ и $b$ и ``заставим'' пройти путь через ребро $e$, то есть сравним изначальное $d[a][b]$ и $d[a][c] + w_e + d[d][b]$ и $d[a][d] + w_e + d[c][b]$, то есть обновляем значение расстояния в матрице в следующем виде: $d[a][b] = \min(d[a][b], d[a][c] + w_e + d[d][b], d[a][d] + w_e + d[c][b])$. Это будет стоить нам $\O(V^2)$ как перебор всех начальных и конечных вершин.
    
    Почему это работает?
    Если раньше оптимальный путь между двумя вершинами не проходил по измененному ребру, то после изменения, потенциально, новый оптимальный путь может проходить через него, поэтому проверим, стал ли потенциальный новый путь меньше, чем исходно оптимальный. Если же новый путь не оптимальнее старого, то ничего менять для таких вершин не надо, так как это изменение не затронуло этот путь (еще стоит отдельно отметить, что изменение этого ребра не затронуло пути от $a$ до $c$ и т.п.). По-хорошему, может случится ситуация, когда вообще все пути начали проходить через это ребро, поэтому надо проверить вообще все вершины.

\end{solution}