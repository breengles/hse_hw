%!TEX root = kotov.tex
\section{Task 3: Из Z в префикс}
\begin{task}
    Преобразовать Z-функцию в префикс-функцию без промежуточного восстановления строки. $\O(n)$.
\end{task}

\begin{solution}
    Переделанное решение, так как не все в этом мире так просто, я не смог сразу осознать, что они ``растут'' в разные стороны, хех.
    В целом, все так же идем по элементам $Z[i]$, пусть в ячейке хранится $k$, смотрим все также в элемент $\pi[i + k - 1]$. Стоит заметить, что есть отношение связь между $Z$ и $\pi$, а именно, что $\pi[i + j] \geq j + 1$, так как по сути это просто длина этого сегмента.

    Присваиваем $\pi[i + k - 1] = j + 1$, и так для всех элементов от $i + k - 1$ до $i$, где $j + 1$ длина соответствующего сегмента. Но тут может быть проблема, что мы перезапишем что-то, что уже было присвоено.
    Посмотрим, почему так делать не стоит: пусть мы куда-то присвоили $x$, пока были на позиции $i$, и пытаемся присвоить значение $y$ с другой позиции $j > i$.
    Тогда заметим, что если мы присваиваем туда же, то $i + x = j + y$, ну тогда $y < x$, значит мы уменьшаем значение, которое было бы в префикс-функции. Таким образом, мы должны будем прерваться, если наткнулись на элемент, в котором уже что-то записано.

    Пока что получилось, что у нас есть цикл внешний, который бежит по всему массиву, а внутренний цикл, кажется, что будет $\O(n^2)$, но это не так, так как на самом деле мы запишем в какую-то ячейку не более одного раза, последующие циклы будут прерываться сразу на этом элементе, конечно, это тоже стоит каких-то денег, но мы в итоге мы присваиваем всего лишь $n$ элементов (это можно представить себе, что если мы закинули удочку, то следующее забрасывание будет проходить как бы из той точки, до куда мы забросили удочку в прошлый раз), при этом прерываний у нас будет не больше, чем $n$, то есть сложность будет что-то типа $\O(n) + n\cdot\O(1) = \O(n)$.
\end{solution}