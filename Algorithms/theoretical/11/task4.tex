%!TEX root = kotov.tex
\section{Task 4: Подпалиндромы}
\begin{task}
    Найти количество подпалиндромов строки. $\O(n\log n)$
\end{task}

\begin{solution}
    Сделаем следующий предподсчет: заведем два как бы разных, но похожим хеша, который я обзову левый и правый хеши:
    \begin{itemize}
        \item Левый хеш будет хешировать префиксы слева-направо, им мы прохешируем все префиксы исходной строки
        \item Правй хеш будет хешировать суффиксы справа-налево, им мы прохешируем все суффиксы исходной строки
    \end{itemize}
    Так как палиндром --- это нечто, что читается слева-направо так же, как и справа-налево, то относительно ``центра симметрии'', если бы мы прохешировали левую и правую части в соответствии с приведенной выше схемой. Это будет стоить нам $\O(n)$

    Теперь, займемся пока без деталей поиском таких палиндромов с помощью хешей: начнем с более понятного, как по мне, случая, когда мы рассматриваем палиндромы нечетной длины, чтобы ``центр симметрии'' был хорошо определен в виде некоторого элемента $x$.
    ``Закинем удочку'' от элемента $x$ как можно дальше (в смысле так далеко, насколько позволят границы исходной строки в обе стороны). Получится некий симметричный отрезок с центром в точке $x$.
    Теперь, благодаря знанию соответствующих хешей на этих отрезках (так как знаем хеши на префикса/суффиксах, то за $\O(1)$ можем вычислять хеш на произвольном отрезке) мы можем быстро отвечать через сравнение хешей левого части с правой на вопрос действительно ли они симметричны.
    Пока мы научились только проверять, но не искать.
    Теперь заведем бинпоиск на самый длинный палиндром с центром в символе $x$, сдвигая в правильную сторону границы симметричного отрезка в зависимости от ответа на предыдущий вопрос.
    Эта процедура стоила нам только что $\O(\log n)$.
    Найдя самый длинный палиндром, по его длине (очень похоже на первую задачку, так как если откусить и слева, и справа по символу, то палиндром останется палиндромом), найдем количество палиндромов с центром в $x$.
    Так пробежимся по всем $x$, в итоге все удовольствие стоило нам $\O(n\log n + n) = \O(n\log n)$.

    Закономерно задать вопрос: а что делать если у нас четные палиндромы? Тут поступим аналогичным образом, только будем рассматривать уже пары соседних $xx$.

\end{solution}

