%!TEX root = kotov.tex
\section{Task 2}
\begin{task}
    Про единственность минимального разреза
\end{task}

\begin{solution}
    \begin{enumerate}[a)]
        \item Сначала найдем минимальный разрез $C(s,t)$, это будет стоить нам по алгоритму Ф-Ф $\O(E|f|)$ (да-да, он нам выдает максимальный поток, но можно и восстановить минимальный разрез, например, запустив два раза \texttt{dfs} из истока и стока по дополняющей сети, получается, так как максимальный поток равен величине минимального разреза, т.е. за $\O(E)$ по потоку восстановить минимальный разрез, что не хуже, чем исходный Ф-Ф).
        Теперь пробежим по всем ребрам найденного минимального разреза и попытаемся увеличить пропускную способность текущего ребра и запустить вновь Ф-Ф (тут вообще говоря спорный вопрос, а за полиномиальное ли время Ф-Ф находит нам поток, так как $|f|\simeq \exp(V,T)$, например). Если каждый раз найденный максимальный поток увеличивался, то исходный разрез --- единственный минимальный, если хотя бы на одном шаге не увеличивался, то это значит, что мы мажорированы каким-то другим минимальный разрезом.
        Вся эта процедура стоила нам $\O(E\cdot E|f|) = \O(E^2|f|)$.

        \item Вообще можно было бы и предыдущий пункт так же сделать: в данном потоке запустим два \texttt{dfs} из истока и стока как в предыдущем пункте, т.е. мы как бы запретим дополнительно \texttt{dfs} ходить по насыщенным ребрам (т.е. найдем какой-то минимальный разрез).
        В итоге мы получим две компоненты связности (с т.з. дополняющей сети), в одной будет содержаться исток, в другой --- сток. Теперь проверим, что сумма мощностей множества вершин в этих двух компонентах равна мощности множества вершин исходного графа.
        Если это так, то разрез единственен, иначе есть еще один минимальный разрез (он или они как бы отделяет(ют) еще одну или несколько других компонент, до которых мы не добрались из $s$ и $t$).

    \end{enumerate}
    
\end{solution}