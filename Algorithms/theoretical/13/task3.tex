%!TEX root = kotov.tex
\section{Task 3: Эратосфен и гладкость}
\begin{solution}
    Заведем массив $d$ размера $n$, в который будем записывать максимальный простой делитель для числа, соответствующего индексу.
    Запустим Эратосфена от $1$ до $n$. По ходу алгоритма, будем записывать в $d$ максимальный делитель для числа, например, сначала мы посмотрим на $2$ и запишем во все индексы, кратные $2$, $2$, затем посмотрим на $3$, и запишем в кратные $3$ --- $3$, т.е. сначала у нас в $6$-ом индексе хранилась $2$, а потом мы обновили и записали туда $3$, это стоило нам $\O(n\log n)$. А еще можно заранее сохранить порядковые номера простых чисел, это нам поможет писать в результирующий массив

    Теперь у нас есть такой массив, так как мы все равно потратили $\O(n\log n)$, то отсортируем этот массив. Теперь будем проходится по этому массиву и в ответный массив, в котором мы будем хранить число $b$-гладких чисел в $b$-ом индексе, записывать количество одинаковых элементов из $d$, плюс количество предыдущего, т.е., например, для $2$ мы просто посчитаем кол-во $2$, для $3$ мы посчитаем кол-во $3$ плюс кол-во $2$, для $4$ --- кол-во $4$ плюс кол-во $3$ (в нем уже записано кол-во $2$) и т.д. Это нужно для того, чтобы учесть тот факт, что если число $2$-гладкое, то оно же и $3$-, $4$- и т.д. гладкое, т.е. его нужно учесть в каждом индексе. Когда мы дойдем вообще до максимального простого делителя среди $n$ чисел, то этот массив от текущего индекса и до конца можно заполнить этим же числом (в которое мы как раз аккумулировали все).
\end{solution}