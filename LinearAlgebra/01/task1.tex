%!TEX root = kotov.tex
\section{Task 1}
Построим матрицу:
\begin{equation}
    \begin{pmatrix}
         2 & 5 & 3 &  0 &  4 \\
        -3 & 2 & 1 & -4 &  9 \\
         1 & 2 & 3 &  2 &  3 \\
         2 & 1 & 0 &  1 & -3 \\
         3 & 2 & 2 &  3 & -2 
    \end{pmatrix}
\end{equation}
Транспонируем:
\begin{equation}
    \begin{pmatrix}
        2 & -3 & 1 &  2 &  3 \\
        5 &  2 & 2 &  1 &  2 \\
        3 &  1 & 3 &  0 &  2 \\
        0 & -4 & 2 &  1 &  3 \\
        4 &  9 & 3 & -3 & -2
    \end{pmatrix}
\end{equation}
Методом Гаусса приводим к ступенчатому виду:
\begin{equation}
    \begin{pmatrix}
        2 &  -3 &  1 & 2 & 3 \\
        0 & -11 & -3 & 6 & 5 \\
        0 &  -4 &  2 & 1 & 3 \\
        0 &   0 &  0 & 0 & 0 \\
        0 &   0 &  0 & 0 & 0
    \end{pmatrix}
\end{equation}
2-ая строчка не зануляется 3-ьей, следовательно, базис:
\begin{equation}
    \left< 
        \begin{pmatrix}
            2\\-3\\1\\2\\3
        \end{pmatrix},
        \begin{pmatrix}
            0\\-11\\-3\\6\\5
        \end{pmatrix},
        \begin{pmatrix}
            0\\-4\\2\\1\\3
        \end{pmatrix}
     \right>
\end{equation}

