%!TEX root = kotov.tex
\section{Task 2}
\subsection{$(a)$}
\begin{gather}
    g = fC \\ f=(1, 1+t,(1+t)^2, (1+t)^3), g = (t^3, t^3-t^2,t^3-t,t^3-1) \\
    \underbrace{\begin{pmatrix}
        0 &  0 &  0 & -1 \\
        0 &  0 & -1 &  0 \\
        0 & -1 &  0 &  0 \\
        1 &  1 &  1 &  1
    \end{pmatrix}}_{G} = 
    \underbrace{
    \begin{pmatrix}
        1 & 1 & 1 & 1 \\
        0 & 1 & 2 & 3 \\
        0 & 0 & 1 & 3 \\
        0 & 0 & 0 & 1
    \end{pmatrix}}_{F}\cdot C \\ \Downarrow \\
    (F|G) \text{ по методу Гаусса получим в левом блоке } I \text{, тогда в правом блоке будет искомая }C \\
    \left(\begin{array}{lccc|rrrr}
        1 & 0 & 0 & 0 & -1 & -2 &  0 & -2 \\
        0 & 1 & 0 & 0 &  3 &  5 &  2 &  3 \\
        0 & 0 & 1 & 0 & -3 & -4 & -3 & -3 \\
        0 & 0 & 0 & 1 &  1 &  1 &  1 &  1
    \end{array}\right)\Longrightarrow 
    C =
    \begin{pmatrix}
        -1 & -2 &  0 & -2 \\
         3 &  5 &  2 &  3 \\
        -3 & -4 & -3 & -3 \\
         1 &  1 &  1 &  1 
    \end{pmatrix}
\end{gather}

\subsection{$(b)$}

Не уверен, что это все надо было доказывать, но мы начнем с отыскания общего вида матрицы поворота относительно произвольного вектора $\vec{n}$ на угол $\theta$. Считаем, что $|\vec{n}| = 1$. Обозначим компоненты матрицы повтора $R$ как $R_{ij}$. Тогда, наша задача --- составить тензор второго ранга из некой комбинации $n_i$, дельта-символа Кронекера $\delta_{ij}$ и символа Леви-Чивиты $\epsilon_{ijk}$. Сделаем это в самом общем виде (подразумеваем соглашение Эйнштейна, то есть по повторяющимся индекса производится суммирование):
\begin{equation}
    R_{ij} = a\delta_{ij} + bn_in_j + c\epsilon_{ijk}n_k
\end{equation}
Так как, логично, что поворот относительно некоторого вектора не должен менять этот самый вектор, то мы можем записать
\begin{gather}
    R_{ij}n_j = n_i \Longrightarrow (a\delta_{ij}+bn_in_j + c\epsilon_{ijk}n_k)n_j = \\
    an_i + bn_i + 0 = n_i \Longrightarrow a + b = 1
\end{gather}
Так как это самый общий вид, то, в частности, это должно работать и для поворота вокруг оси $oZ$. Воспользуемся этим, чтобы определить наши константы (ремарка, эти константы действительно скаляры, причем, они зависят только от угла, так как исходно из скалярных величин у нас только величина угла поворота)
\begin{gather}
    R(\vec{z}, \theta)_{11} = \cos\theta = a \Longrightarrow b = 1 - \cos\theta \\
    R(\vec{z}, \theta)_{12} = -\sin\theta = c
\end{gather}
Таким образом, мы имеем (формула Родриго?)
\begin{equation}
    R_{ij}(\vec{n}, \theta) = \cos\theta\delta_{ij} + (1-\cos\theta)n_in_j-\sin\theta\epsilon_{ijk}n_k
\end{equation}

Теперь к задаче. В нашем случае $\theta = \pi$:
\begin{equation}
    R(\vec{n},\pi)_{ij} = 2n_in_j-\delta_{ij}
\end{equation}
Вектор $\vec{n}$ задан уравнением $\frac{x}{1}=\frac{y}{2}=\frac{z}{-1}$, ему, например, удовлетворяет
\begin{gather}
    \begin{pmatrix}
        1\\2\\-1
    \end{pmatrix} = [\text{нормируем}] = 
    \frac{\sqrt{6}}{6}
    \begin{pmatrix}
        1 \\ 2 \\ -1
    \end{pmatrix}
\end{gather}
P.S. формально, можно было бы и не нормировать.

Вычислим элементы матрицы поворота:
\begin{equation}
    R = 
    \begin{pmatrix}
         -2 &  2 & -1 \\
          2 &  1 & -2 \\
         -1 & -2 & -2
    \end{pmatrix}
\end{equation}
И, если я правильно все понимаю, то это именно то, что нам нужно.
