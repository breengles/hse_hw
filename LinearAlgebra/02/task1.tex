%!TEX root = kotov.tex
\section{Task 1}
\subsection{Случай $p\ne 5,\,7$}
В таком случае, у матрицы, составленной из исходных векторов в строчках, существует обратная матрица:
\begin{equation}
    \begin{pmatrix}
        2 & 1 & 3 \\
        -1 & 2 & 1 \\
        3 & 3 & -1
    \end{pmatrix}^{-1} = 
    \begin{pmatrix}
        \frac17 & -\frac{2}{7} & \frac17 \\
        -\frac{2}{7\cdot 5} & \frac{11}{7\cdot 5} & \frac{1}{7} \\
        \frac{9}{7\cdot 5} & \frac{3}{7\cdot 5} & -\frac{1}{7}
    \end{pmatrix}
\end{equation}
следовательно, существует единственное такое отображение $a$.

\subsection{Случай $p = 5$}
В таком случае, можно заметить, что 
\begin{gather}
    \begin{pmatrix}
        3 \\ 1 \\ -1
    \end{pmatrix} = 3 \cdot
    \begin{pmatrix}
        1 \\ 2 \\ 3
    \end{pmatrix}
\end{gather}
а при этом
\begin{gather}
    a\begin{pmatrix}
        3 \\ 1 \\ -1
    \end{pmatrix} = 
    \begin{pmatrix}
        2 \\ 1 \\0
    \end{pmatrix}\text{, но }
    a\begin{pmatrix}
        1 \\ 2 \\ 3
    \end{pmatrix} = 
    \begin{pmatrix}
        1 \\ 0 \\ 1
    \end{pmatrix} \neq 3\cdot
    \begin{pmatrix}
        1 \\ 0 \\ 1
    \end{pmatrix},
\end{gather}
следовательно, для $p = 5$ таких операторов $a$ не существует.

\subsection{Случай $p = 7$}
Рассмотрим вектора
\begin{gather}
    \begin{pmatrix}
        3 \\ 1 \\ -1
    \end{pmatrix} = 
    \begin{pmatrix}
        2 \\ -1 \\ 3
    \end{pmatrix} + 
    \begin{pmatrix}
        1 \\ 2 \\ 3
    \end{pmatrix}
\end{gather}
при этом они корректно переводятся оператором $a$. То есть, мы знаем, как оператор $a$ переводит два линейно-независимых вектора в два линейно-независимых вектора (перевели одно двумерное пространство в двумерное пространство). У нас осталась свобода в том, куда переводится оставшееся одномерное пространство. Его можно перевести во что угодно, в любой вектор из исходного множества, мощность которого $p^n = 7^3 = 343$, то есть существует $343$ таких операторов $a$.