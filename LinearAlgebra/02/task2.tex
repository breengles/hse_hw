%!TEX root = kotov.tex
\section{Task 2}
Будем пользоваться формулой преобразования матрицы оператора при переходе от одного базиса к другому:
\begin{equation}
    A = DA'C^{-1}
\end{equation}
После того, как мы убедились, что исходные и конечные вектора образуют линейно-независимые системы векторов в соответствующих пространствах, найдем матрицу перехода от этих ``базисов'' к стандартным:
\begin{gather}
    \begin{pmatrix}
        1 &  1 & -1 \\
        1 &  1 &  2 \\
        1 & -1 &  2 
    \end{pmatrix} = 
    \begin{pmatrix}
        1 & 0 & 0 \\
        0 & 1 & 0 \\
        0 & 0 & 1
    \end{pmatrix}C\Longrightarrow C = 
    \begin{pmatrix}
        1 &  1 & -1 \\
        1 &  1 &  2 \\
        1 & -1 &  2
    \end{pmatrix}
\end{gather}
Аналогично поступим с базисом в результирующем пространстве, разве что добавим в кучу четвертый вектор из стандартных (ранг матрицы слева в следующем уравнении как раз соответствует этому, ее ранг равен $4$):
\begin{gather}
    \begin{pmatrix}
        1 & 2 & 1 & 0 \\
        2 & 1 & 0 & 0 \\
        1 & 0 & 1 & 0 \\
        1 & 0 & 1 & 1 
    \end{pmatrix} = ID \Longrightarrow D = 
    \begin{pmatrix}
        1 & 2 & 1 & 0 \\
        2 & 1 & 0 & 0 \\
        1 & 0 & 1 & 0 \\
        1 & 0 & 1 & 1 
    \end{pmatrix}
\end{gather}
Итого, получаем:
\begin{gather}
    A = DA'C^{-1} = 
    \begin{pmatrix}
        1 & 2 & 1 & 0 \\
        2 & 1 & 0 & 0 \\
        1 & 0 & 1 & 0 \\
        1 & 0 & 1 & 1 
    \end{pmatrix}
    \begin{pmatrix}
        1 & 0 & 0 \\
        0 & 1 & 0 \\
        0 & 0 & 1 \\
        0 & 0 & 0
    \end{pmatrix} \frac{1}{6}
    \begin{pmatrix}
         4 & -1 &  3 \\
         0 &  3 & -3 \\
        -2 &  2 &  0
    \end{pmatrix} = \frac{1}{6}
    \begin{pmatrix}
        2 & 7 & -3 \\
        8 & 1 &  3 \\
        2 & 1 &  3 \\
        2 & 1 &  3 
    \end{pmatrix}
\end{gather}
Проверим:
\begin{gather}
    \frac16
    \begin{pmatrix}
        2 & 7 & -3 \\
        8 & 1 &  3 \\
        2 & 1 &  3 \\
        2 & 1 &  3 
    \end{pmatrix}
    \begin{pmatrix}
        1 \\ 1 \\ 1
    \end{pmatrix} = 
    \begin{pmatrix}
        1 \\ 2 \\ 1 \\ 1
    \end{pmatrix} \\
    \frac16
    \begin{pmatrix}
        2 & 7 & -3 \\
        8 & 1 &  3 \\
        2 & 1 &  3 \\
        2 & 1 &  3 
    \end{pmatrix}
    \begin{pmatrix}
        1 \\ 1 \\ -1
    \end{pmatrix} = 
    \begin{pmatrix}
        2 \\ 1 \\ 0 \\ 0
    \end{pmatrix} \\
    \frac16
    \begin{pmatrix}
        2 & 7 & -3 \\
        8 & 1 &  3 \\
        2 & 1 &  3 \\
        2 & 1 &  3 
    \end{pmatrix}
    \begin{pmatrix}
        -1 \\ 2 \\ 2
    \end{pmatrix} = 
    \begin{pmatrix}
        1 \\ 0 \\ 1 \\ 1
    \end{pmatrix} \\
\end{gather}
Таким образом, искомое матричное представление в стандартном базисе:
\begin{equation}
    A = 
    \begin{pmatrix}
        2 & 7 & -3 \\
        8 & 1 &  3 \\
        2 & 1 &  3 \\
        2 & 1 &  3 
    \end{pmatrix}
\end{equation}