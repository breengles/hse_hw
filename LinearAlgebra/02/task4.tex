%!TEX root = kotov.tex
\section{Task 4 (индивидуальное 2.6)}
Будем рассматривать матрицу нашего оператора в стандартном базисе. Тогда можно будет рассмотреть переход от этого базиса, к интересующему нас базису $f$.

Найдем матрицу $A_e$ такую, что
\begin{gather}
    A_e
    \begin{pmatrix}
        1 & 1 & 2 \\
        2 & 1 & 0 \\
        1 & 1 & 1
    \end{pmatrix} = 
    \begin{pmatrix}
        -7 & -4 &  1 \\
        -5 & -3 &  1 \\
         5 &  3 & -1
    \end{pmatrix}
\end{gather}
Транспонируем левую и правую части:
\begin{gather}
    \begin{pmatrix}
        1 & 2 & 1 \\
        1 & 1 & 1 \\
        2 & 0 & 1
    \end{pmatrix} A_e^T = 
    \begin{pmatrix}
        -7 & -5 &  5 \\
        -4 & -3 &  3 \\
         1 &  1 & -1
    \end{pmatrix} \Longrightarrow A_e = 
    \begin{pmatrix}
         2 & -3 & -3 \\
         2 & -2 & -3 \\
        -2 &  2 &  3
    \end{pmatrix}
\end{gather}
Теперь найдем матрицы перехода:
\begin{gather}
    [f] = [e]C \Longrightarrow
    \begin{pmatrix}
        1 & 1 & 2 \\
        1 & 0 & 1 \\
        0 & 1 & 0
    \end{pmatrix} = 
    \begin{pmatrix}
        1 & 0 & 0 \\
        0 & 1 & 0 \\
        0 & 0 & 1    
    \end{pmatrix} C \Longrightarrow
    C = 
    \begin{pmatrix}
        1 & 1 & 2 \\
        1 & 0 & 1 \\
        0 & 1 & 0
    \end{pmatrix}
\end{gather}
Найдем еще обратную к $C$:
\begin{equation}
    C^{-1} = 
    \begin{pmatrix}
        -1 &  2 &  1 \\
         0 &  0 &  1 \\ 
         1 & -1 & -1
    \end{pmatrix}.
\end{equation}
Тогда
\begin{gather}
    A_f = C^{-1}A_eC = 
    \begin{pmatrix}
        -1 &  2 &  1 \\
        0 &  0 &  1 \\ 
        1 & -1 & -1
    \end{pmatrix}
    \begin{pmatrix}
        2 & -3 & -3 \\
        2 & -2 & -3 \\
       -2 &  2 &  3
    \end{pmatrix}
    \begin{pmatrix}
        1 & 1 & 2 \\
        1 & 0 & 1 \\ 
        0 & 1 & 0
    \end{pmatrix} = 
    \begin{pmatrix}
         1 &  0 &  1 \\
         0 &  1 & -2 \\
        -1 & -1 &  1
    \end{pmatrix}
\end{gather}
