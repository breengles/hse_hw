%!TEX root = kotov.tex
\section{Task 3}
\begin{gather}
    U_1 = \{a| [a,
    \begin{pmatrix}
        1 & 2 \\
        2 & 1    
    \end{pmatrix}] = 0\}\quad 
    U_2 = \{b| [b,
    \begin{pmatrix}
        1 & 1 \\
        0 & 1    
    \end{pmatrix}] = 0\}
\end{gather}

Для начала найдем базисы для $U_1$ и $U_2$:
\begin{itemize}
    \item $U_1$:
    \begin{gather}
        \begin{pmatrix}
            a_{11} & a_{12} \\
            a_{21} & a_{22} 
        \end{pmatrix}
        \begin{pmatrix}
            1 & 2 \\
            2 & 1
        \end{pmatrix} =
        \begin{pmatrix}
            a_{11} + 2a_{12} & 2a_{11} + a_{12} \\
            a_{21} + 2a_{22} & 2a_{21} + a_{22}
        \end{pmatrix} = \\
        \begin{pmatrix}
            1 & 2 \\
            2 & 1
        \end{pmatrix}
        \begin{pmatrix}
            a_{11} & a_{12} \\
            a_{21} & a_{22} 
        \end{pmatrix} = 
        \begin{pmatrix}
            a_{11} + 2a_{21} & a_{12} + 2a_{22} \\
            2a_{11} + a_{21} & 2a_{12} + a_{22}
        \end{pmatrix}
    \end{gather}
    Рассмотрим компоненту $(1,1):a_{11}+2a_{12} = a_{11} + 2a_{21} \Longrightarrow a_{12} = a_{21} = y$
    
    и компоненту $(2,1):a_{21}+2a_{22} = 2a_{11} + a_{21} \Longrightarrow a_{11} = a_{22} = x$, что дает
    \begin{eqnarray}
        a = 
        \begin{pmatrix}
            x & y \\
            y & x    
        \end{pmatrix}
    \end{eqnarray}
    
    Таким образом 
    \begin{gather}
        U_1 =
        \left<
            \begin{pmatrix}
                1 & 0 \\
                0 & 1
            \end{pmatrix},
            \begin{pmatrix}
                0 & 1 \\
                1 & 0
            \end{pmatrix}
        \right>
    \end{gather}
    \begin{remark}
        Вообще говоря, не удивительно, что мы имеем единичную матрицу в качестве одно из базисных векторов, так как единичная матрица коммутирует с любыми матрицами.
    \end{remark}
    \item $U_2$:
    \begin{gather}
        \begin{pmatrix}
            b_{11} & b_{12} \\
            b_{21} & b_{22}
        \end{pmatrix}
        \begin{pmatrix}
            1 & 1 \\
            0 & 1
        \end{pmatrix} =
        \begin{pmatrix}
            b_{11} & b_{11} + b_{12} \\
            b_{21} & b_{21} + b_{22}
        \end{pmatrix} =
        \begin{pmatrix}
            1 & 1 \\
            0 & 1
        \end{pmatrix}
        \begin{pmatrix}
            b_{11} & b_{12} \\
            b_{21} & b_{22} 
        \end{pmatrix} =
        \begin{pmatrix}
            b_{11} + b_{21} & b_{12} + b_{22} \\
            b_{21} & b_{22}
        \end{pmatrix}
    \end{gather}
    Рассмотрим компоненты $(1, 1):b_{11} = b_{11} + b_{21} \Longrightarrow b_{21} = 0$, и компоненты $(1,2): b_{11} + b_{12} = b_{12} + b_{22} \Longrightarrow b_{11}=b_{22}=x$, свободной осталась компоненты $b_{12}$
    Таким образом
    \begin{gather}
        U_2 =
        \left<
            \begin{pmatrix}
                1 & 0 \\
                0 & 1
            \end{pmatrix},
            \begin{pmatrix}
                0 & 1 \\
                0 & 0
            \end{pmatrix}
        \right>
    \end{gather}
\end{itemize}

\subsection{$U_1 + U_2$)}
\begin{gather}
    U_1 + U_2 =
    \left<
        \begin{pmatrix}
            1 & 0 \\
            0 & 1
        \end{pmatrix},
        \begin{pmatrix}
            0 & 1 \\
            1 & 0
        \end{pmatrix}
    \right> + 
    \left<
        \begin{pmatrix}
            1 & 0 \\
            0 & 1
        \end{pmatrix},
        \begin{pmatrix}
            0 & 1 \\
            0 & 0
        \end{pmatrix}
    \right> = 
    \left<
        \begin{pmatrix}
            1 & 0 \\
            0 & 1
        \end{pmatrix},
        \begin{pmatrix}
            0 & 1 \\
            1 & 0
        \end{pmatrix},
        \begin{pmatrix}
            0 & 1 \\
            0 & 0
        \end{pmatrix}
    \right>
\end{gather}
\begin{remark}
    Так как вторая матрица из линейной оболочки порождает нам матрицы только с одинаковыми элементами на побочной диагонали, а в $U_2$ содержатся матрицы с различными элементами на побочной диагонали, то
    \begin{gather}
        \begin{pmatrix}
            0 & 1 \\
            1 & 0
        \end{pmatrix},
        \begin{pmatrix}
            0 & 1 \\
            0 & 0
        \end{pmatrix}
    \end{gather}
    линейно независимы.
\end{remark}

\subsection{$U_1 \cap U_2$)}
\begin{gather}
    U_1 \cap U_2 = 
    \left<
        \begin{pmatrix}
            1 & 0 \\
            0 & 1
        \end{pmatrix},
        \begin{pmatrix}
            0 & 1 \\
            1 & 0
        \end{pmatrix}
    \right>\cap
    \left<
        \begin{pmatrix}
            1 & 0 \\
            0 & 1
        \end{pmatrix},
        \begin{pmatrix}
            0 & 1 \\
            0 & 0
        \end{pmatrix}
    \right> = 
    \left<
        \begin{pmatrix}
            1 & 0 \\
            0 & 1
        \end{pmatrix}
    \right> = 
    \left<
        \begin{pmatrix}
            1 & 0 \\
            0 & 1
        \end{pmatrix}
    \right>
\end{gather}
\begin{remark}
    Так как вторая линейная оболочка из пересечения порождает матрицы только с нулевой $(2,1)$-компонентой, то для того, чтобы матрица из первой оболочки также была в данном подпространстве, то необходимо, чтобы коэффициент в линейной комбинации 2-ой матрицы из 
    $ \begin{pmatrix}
            1 & 0 \\
            0 & 1
        \end{pmatrix},
        \begin{pmatrix}
            0 & 1 \\
            1 & 0
        \end{pmatrix}$ всегда был равен нулю.
\end{remark}
\begin{remark}
    Да, можно было бы это все проделать с помощью процедуры Гаусса, взяв за стандартный базис 
    \begin{gather}
        \begin{pmatrix}
            1 & 0 \\ 0 & 0
        \end{pmatrix},
        \begin{pmatrix}
            0 & 1 \\ 0 & 0
        \end{pmatrix},
        \begin{pmatrix}
            0 & 0 \\ 1 & 0
        \end{pmatrix},
        \begin{pmatrix}
            0 & 0 \\ 0 & 1
        \end{pmatrix}
    \end{gather}
    и рассматривать матрицы как вектора в этом базисе, то есть
    \begin{gather}
        \begin{pmatrix}
            1 & 0 \\
            0 & 1
        \end{pmatrix} \Leftrightarrow 
        \begin{pmatrix}
            1 \\ 0 \\ 0 \\ 1
        \end{pmatrix}
    \end{gather}.
\end{remark}
\subsection{Второй способ?}
Введем векторное представление для матриц в стандартном базисе :
\begin{gather}
    \begin{pmatrix}
        1 & 0 \\
        0 & 0
    \end{pmatrix}
    \begin{pmatrix}
        0 & 1 \\
        0 & 0
    \end{pmatrix}
    \begin{pmatrix}
        0 & 0 \\
        1 & 0
    \end{pmatrix}
    \begin{pmatrix}
        0 & 0 \\
        0 & 1
    \end{pmatrix}
\end{gather}

Определим, какой СЛАУ задается пространство $U_1$. Для этого найдем базис решений следующей СЛАУ:
\begin{equation}
    \begin{pmatrix}
        1 & 0 & 0 & 1 \\
        0 & 1 & 1 & 0
    \end{pmatrix}
\end{equation}
Он, например, может выглядеть так (уложены в строчки):
\begin{equation}
    \begin{pmatrix}
        1 & 0 &  0 & -1 \\
        0 & 1 & -1 &  0
    \end{pmatrix} = A_1.
\end{equation}
Аналогично, получим СЛАУ для $U_2$:
\begin{equation}
    A_2 = 
    \begin{pmatrix}
        1 & 0 & 0 & -1 \\
        0 & 0 & 1 &  0
    \end{pmatrix}
\end{equation}
Для того, чтобы найти базис в пересечении подпространств, нам надо найти базис решения совместной СЛАУ, матрица которой:
\begin{equation}
    \begin{pmatrix}
        A_1 \\
        A_2
    \end{pmatrix} = 
    \begin{pmatrix}
        1 & 0 &  0 & -1 \\
        0 & 1 & -1 &  0 \\
        1 & 0 &  0 & -1 \\
        0 & 0 &  1 &  0
    \end{pmatrix} \underbrace{\rightarrow}_{\text{Гаусс}}
    \begin{pmatrix}
        1 & 0 & 0 & -1 \\
        0 & 1 & 0 &  0 \\
        0 & 0 & 1 &  0 
    \end{pmatrix}
\end{equation}
Из последнего видно, что $x_1 = x_4, \, x_2 = 0, \, x_3 = 0$, следовательно, базис будет состоять из одного вектора, например, $(1,\,0,\,0,\,1)^T \Longleftrightarrow
\begin{pmatrix}
    1 & 0 \\
    0 & 1
\end{pmatrix}$

