%!TEX root = kotov.tex
\section{Task 2}
\begin{task}
    Матрица оператора имеет вид $$A =\begin{pmatrix}-3&-4&4\\-2&-5&4\\-4&-8&7 \end{pmatrix}.$$ Является ли оператор диагонализуемым?  В случае положительного ответа найдите базис, в котором матрица оператора диагональна и матрицу оператора в этом базисе.
\end{task}

\begin{solution}
    \begin{gather}
        \det(A-\lambda I) = 
        \begin{Vmatrix}
            -3-\lambda & -4 & 4 \\
            -2 & -5-\lambda & 4 \\
            -4 & -8 & 7-\lambda
        \end{Vmatrix} = -\lambda^3-\lambda^2+\lambda+1 = (\lambda-1)(\lambda+1)^2 = 0
    \end{gather}
    Таким образом, получили $\lambda_1 = 1$ --- не вырожденный корень и $\lambda_2 = -1$ --- двукратно вырожденный корень.
    
    Посмотрим собственные вектора:
    \begin{itemize}
        \item
        \begin{gather}
            \lambda = 1:\quad
            \begin{pmatrix}
                -4 & -4 & 4 \\
                -2 & -6 & 4 \\
                -4 & -8 & 6
            \end{pmatrix} \Longrightarrow \vec{v}_1 = 
            \begin{pmatrix}
                0.5 \\ 0.5 \\ 1
            \end{pmatrix}
        \end{gather}
        \item
        \begin{gather}
            \lambda = -1: \quad 
            \begin{pmatrix}
                -2 & -4 & 4 \\
                -2 & -4 & 4 \\
                -4 & -8 & 8
            \end{pmatrix} \Longrightarrow \text{ ФСР } = \left\{
                a
                \begin{pmatrix}
                    -2 \\ 1 \\ 0
                \end{pmatrix} + b
                \begin{pmatrix}
                    2 \\ 0 \\ 1
                \end{pmatrix}
            \right\} \Longrightarrow \vec{v}_2 =
            \begin{pmatrix}
                -2 \\ 1 \\ 0
            \end{pmatrix}, \quad \vec{v}_3 = 
            \begin{pmatrix}
                2 \\ 0 \\ 1
            \end{pmatrix}
        \end{gather}
    \end{itemize}
    Таким образом, в базисе $\{\vec{v}_1,\vec{v}_2,\vec{v}_3\}$
    \begin{eqnarray}
        A = 
        \begin{pmatrix}
            1 &  0 &  0 \\
            0 & -1 &  0 \\
            0 &  0 & -1
        \end{pmatrix}
    \end{eqnarray}
\end{solution}