%!TEX root = kotov.tex
\section{Task 1}
\begin{task}
    Докажите, что любую матрицу ранга $r$ можно представить в виде суммы $r$ матриц ранга единица, но нельзя представить в виде суммы менее чем $r$ таких матриц.
\end{task}

\begin{solution}
    Рассмотрим набор строчек исходной матрицы $A$: выберем в нем некоторый базис (по условию $\rk A = r$, следовательно размер этого базиса будет также $r$), остальные же строчки разложим по этому базису, обозначим коэффициент разложение $k$-ой строки по $i$-ой как $c^i_{k}$.
    Пусть ищем разложение следующего вида $A = \sum\limits_{i=1}^r A_i$, в качестве $A_i$ будем выбирать следующую структуру: поместим на $i$-ую строку соответствующую строку исходной матрицы, которая является одной из базисной (обозначим ее за $a_i$), а на $k-ую$ позицию строку $c^i_k a_i$, то есть эти строки будут иметь вид исходной базисной строки, умноженных на соответствующий коэффициент разложения этих же исходных строк с участием данной базисной строки.
    Так мы получим какой-то набор одноранговых (так как в них лишь одна строка линейно-независимы) матриц, сумма которых равна исходной матрицы $A$.

    Теперь поймем, почему меньше, чем $r$ штук одноранговых матриц нельзя.
    Воспользуемся теоремой $\rk (\sum A_i) \leq \sum \rk A_i$, то есть ранг суммы матрицы не превосходит суммы рангов матриц, тогда, если бы справа в этом неравенстве было бы меньше, чем $r$ слагаемых, то слева было бы уже точно не больше, поэтому меньше, чем $r$ штук нельзя
\end{solution}