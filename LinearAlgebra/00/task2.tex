%!TEX root = kotov.tex
\section{Task 2}
Рассмотрим СЛАУ на коэффициенты для системы уравнений, задающую $V$ (сразу переставим строчки более удобным образом):
\begin{equation}
    \begin{pmatrix}
        1 & 0 & 0 & 0 & 1 & 1 \\
        1 & 1 & 0 & 0 & 0 & 1 \\
        1 & 1 & 1 & 0 & 0 & 0 \\
        0 & 0 & 0 & 1 & 1 & 1
    \end{pmatrix}
\end{equation}
После преобразования матрицы методом Гаусса получим:
\begin{equation}
    \begin{pmatrix}
        1 & 0 & 0 & 0 &  1 & 1 \\
        0 & 1 & 0 & 0 & -1 & 0 \\
        0 & 0 & 1 & 0 &  0 & -1 \\
        0 & 0 & 0 & 1 &  1 & 1
    \end{pmatrix}
\end{equation}
Что приводит к
\begin{gather}
    \begin{cases}
        a_4 = -a_5 - a_6 \\
        a_3 = a_6 \\
        a_2 = a_5 \\
        a_1 = -a_5 - a_6
    \end{cases} \Longrightarrow
    \begin{cases}
        a_1 = -a_5 - a_6 \\
        a_2 = a_5 \\
        a_3 = a_6 \\
        a_4 = -a_5 - a_6
    \end{cases} \Longrightarrow a_5, a_6 \text{ --- свободные параметры}
\end{gather}
Выберем сначала $a_5 = 1, a_6 = 0$, а затем $a_5 = 0, a_6 = 1$, тогда получим два уравнения из системы:
\begin{equation}
    \begin{cases}
        -x_1 + x_2 - x_4 + x_5 = 0 \\
        -x_1 + x_3 - x_4 + x_6 = 0
    \end{cases}
\end{equation}
