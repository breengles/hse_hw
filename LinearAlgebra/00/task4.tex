%!TEX root = kotov.tex
\section{Task 4}
\subsection{$\{\vec{v}_i + \vec{v}_j\}_{i \neq j}$}
% Без ограничения общности будем считать, что $\{\vec{v_i}\}$ --- ортонормированный базис.
Для $\{\vec{v}_i + \vec{v}_j\}_{i \neq j} = \hat{V}$ предъявим схему, по которой можно выбрать базис из $\hat{V}$ в $V$:
Рассмотрим $\{\vec{v}_1 + \vec{v}_j\}_{j = 2}^n$.
Это множество содержит $n - 1$ линейно-независимых векторов, их координаты к исходном базисе:
\begin{equation}
    \begin{pmatrix}
        1 \\
        1 \\
        0 \\
        0 \\
        \vdots \\
        0
    \end{pmatrix}
    \begin{pmatrix}
        1 \\
        0 \\
        1 \\
        0 \\
        \vdots \\
        0
    \end{pmatrix} \ldots
    \begin{pmatrix}
        1 \\
        0 \\
        0 \\
        \vdots \\
        0 \\
        1
    \end{pmatrix}
\end{equation}
Из их представления в базисе $\{\vec{v}_i\}$ как раз наглядно видна их линейная независимость, но их лишь $n - 1$ штук.

Добавим $\vec{u} = (0, 1, 1, 0, \ldots, 0)^T$ в этот набор. Тогда, надо проверить, что мы не нарушили линейную независимость.
Рассмотрим следующую тройку (остальные вектора содержат вторую $1$ месте с б\`ольшим индексом, следовательно, ``новый'' вектор не скажется на них, а они ни при каких коэффициентах не окажут влияния на линейную зависимость/независимость рассматриваемой тройки, хотя формально можно было бы посчитать определитель матрицы, составленный из всех этих векторов и убедиться, что он не равен $0$):

\begin{equation}
    \begin{pmatrix}
        1 \\ 1 \\ 0 \\ 0 \\ \vdots \\ 0
    \end{pmatrix}
    \begin{pmatrix}
        1 \\ 0 \\ 1 \\ 0 \\ \vdots \\ 0
    \end{pmatrix}
    \begin{pmatrix}
        0 \\ 1 \\ 1 \\ 0 \\ \vdots \\ 0
    \end{pmatrix}
\end{equation}
Если не учитывать ``несущественные'' нули далее 3-ей компоненты, то можно заметить, что эта система векторов есть попарные суммы стандартных базисных векторов в $D3$, которые линейно-независимы. Таким образом, можно выбрать набор $\{\{\vec{v}_1 + \vec{v}_j\}^{n}_{j = 2}, \vec{u}\}$.

\subsubsection{Случай поля $1+1=0 \, (1 = -1)$}
\label{sec:char2}
Рассмотрим вектора $\vec{w}_{ij} = \vec{v}_i + \vec{v}_j$.
Пусть $\vec{b}_i = \vec{v}_1 + \vec{v}_i$ по прежнему будем рассматриваться нами как часть искомого базиса (при другом возможном наборе $n-1$ векторов можно переразложить их по $\{\vec{b}_i\}$).
Тогда, для того, чтобы искомый набор являлся базисов в исходном пространстве $V$, достаточно добавить еще один вектор, который будет линейно-независим с $\{\vec{b}_i\}$.
Покажем, что нельзя выбрать такой $\vec{w}_{ij}$:
\begin{gather}
    \vec{w}_{ij}\big|_{i \neq 1} = \vec{v}_i + \vec{v}_j = (\vec{v}_1 - \vec{v}_1) + \vec{v}_i + \vec{v}_j = \vec{v}_1 + \vec{v}_1 + \vec{v}_i + \vec{v}_j = 
    (\vec{v}_1 + \vec{v}_i) + (\vec{v}_1 + \vec{v}_j) = \vec{b}_i + \vec{b}_j
\end{gather}
То есть любой оставшийся вектор $\vec{w}_{ij}$ является линейной комбинацией $\{\vec{b}_i\} \Longrightarrow$ нельзя выбрать базис в таком пространстве.

\subsection{$\{\vec{v}_i - \vec{v}_j\}_{i \neq j}$}
Заметим, что конечномерные линейные пространства размерности $n$ изоморфны $\mathbb{R}^n$, которое в свою очередь представляет в виде прямой суммы $\mathbb{R}^n = F \bigoplus G$, где $F = \{\vec{x} \in \mathbb{R}^n : \sum x^i = 0\}$ и $G = \{\vec{x} \in \mathbb{R}^n : x^1 = x^2 = \ldots = x^n\}$.
Можем рассмотреть $\{\vec{v}_i - \vec{v}_j\}_{i \neq j}$ как подмножество $U = \{\vec{x} : \sum x^i = 0\}$. Но тогда видно, что нам не хватает, как минимум, вектора из $G$, то есть из $\{\vec{v}_i - \vec{v_j}\}_{i \neq j}$ можно выбрать в лучшем случае базис только для $F$.

\subsubsection{Случай поля 1+1 = 0 \, (1 = -1)}
$\vec{u}_{ij} = \vec{v}_i - \vec{v}_j = \vec{v}_i + \vec{v}_j = \vec{w}_{ij} \Longrightarrow$ можно сослаться на \ref{sec:char2}.