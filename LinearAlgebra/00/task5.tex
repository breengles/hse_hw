%!TEX root = kotov.tex
\section{Task 5}
\subsection{Отсутствие одно-элементных подмножеств}
Воспользуемся изоморфизмом $2^M \longleftrightarrow ^{n}K : x^i \in {0, 1}$, то есть с бинарными строками длины $n$.
В терминах бинарных строк, утверждается, что можно выбрать такое подпространство в $2^M$ размерности $n - 1$, что ни какой элемент не будет содержать ровно одну $1$.
Рассмотрим следующие бинарные строки:
\begin{align}
    1 & 1 0   \ldots 0 \\
    1 & 0 1 0 \ldots 0 \\
    \vdots \\
    1 & 0 \ldots 0 1
\end{align}
Количество таких строк $n - 1$, они линейно-независимы.
Видно, что сумма любой пары этих строк даст строку также содержащую две $1$, так как $x^1$ сократится, а на остальных координатах стоят непересекающиеся $0$ и $1$. То есть в линейных комбинациях не будет содержаться строки с лишь одной $1$.

\subsection{Наличие двух-элементных подмножеств}
Рассмотрим следующий набор бинарных строк:
\begin{align}
    1 & 0   \ldots 0 \\
    0 & 1 0 \ldots 0 \\
    \vdots \\
    0 & \ldots 0 1
\end{align}
Очевидно, что это базис (причем, стандартный) в $n$-мерном пространстве.
Выкинем какой-то вектор из этого набора, тогда останется базис $n-1$ мерного пространства.
Видно, что в нем всегда будут получатся в линейных комбинациях строки, содержащие две $1$.

Но это был какой-то конкретный базис. Предположим, что нашелся другой базис, тогда мы переразложим найденный базис, по стандартному.
