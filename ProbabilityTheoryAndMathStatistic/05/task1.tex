%!TEX root = kotov.tex
\section{Task}
\begin{task}
    Про эффективность оценки для выборки из дискретного равномерного распределения.
\end{task}

\begin{solution}
    Будем использовать теорему Лемана--Шеффе, но для этого сначала покажем несмещенность оценки:
    \begin{equation}
        S = \frac{X^{n + 1}_{(n)} - (X_{(n)} - 1)^{n + 1}}{X^{n}_{(n)} - (X_{(n)} - 1)^{n}}.
    \end{equation}
    Для этого посчитаем сначала вероятность $P(X_{(n)} = k) = \sum\limits_{j=1}^{n}{n \choose j}(\frac{1}{\theta})^j(\frac{k - 1}{\theta})^{n - j} = \frac{k^n - (k - 1)^n}{\theta^n}$ (а-ля сумма вероятностей иметь $j$ максимумов иметь значение $k$, а остальные имеют значение меньше, чем $k$).

    Рассмотрим матожидание $S$
    \begin{gather}
        ES = \sum\limits_{k = 1}^{\theta}\frac{k^n - (k - 1)^{n}}{\theta^n}\frac{k^{n+1} - (k-1)^{n+1}}{k^n - (k - 1)^{n}} = \sum\limits_{k = 1}^{\theta}\frac{k^{n+1} - (k - 1)^{n + 1}}{\theta^n}
    \end{gather}
    В последней сумме заметим, что второй вклад в числителе будет сокращать первый для предыдущего члена суммы, следовательно, выживет только первый вклад числителя для последнего члена суммы, т.е. $\frac{theta^{n + 1}}{\theta^n} = \theta$.
    Таким образом, оценка действительно несмещенная.

    Теперь разберемся со статистикой: возьмем в качестве кандидата на подходящую статистку для теоремы Лемана--Шеффе, бросающийся в глаза $T = X_{(n)}$ (вообще можно было бы получить ее из факторизации совместной плотности).
    Рассмотрим плотность совместного распределения 
    \begin{gather}
        P(X_1= k_1, \ldots, X_n = k_n) = \underbrace{\frac{1}{\theta^n}1[X_{(n)} \leq \theta]}_{g_{\theta}} \underbrace{1[X_{(1)} \geq 1]}_{h}.
    \end{gather}
    Т.е. $T$ --- достаточная статистика.

    Теперь покажем полноту $T$:
    \begin{gather}
        0 = E_{\theta}\phi(T) = \sum\limits_{k = 1}^{\theta}\phi(k)\frac{k^n - (k - 1)^n}{\theta^n}
    \end{gather}
    Так как это верно для $\forall \theta \in \mathbb{N}$, то в частности верно и для $\theta = 1$:
    \begin{equation}
        \phi(1) \cdot \text{const} = 0 \Longrightarrow \phi(1) = 0
    \end{equation}
    и для $\theta = 2$:
    \begin{equation}
        \underbrace{\phi(1)}_{=0}\cdot\ldots + \phi(2)\cdot\text{const} = 0 \Longrightarrow \phi(2) = 0
    \end{equation}
    и так далее по такой цепочке получим, что $\phi(k) = 0$ $\forall k \in \mathbb{N}$, то есть $T$ --- полная статистика.

    Таким образом, у нас в руках сейчас есть некоторая несмещенная оценка $S(T)$ и полная и достаточная статистика $T$, следовательно по теореме Лемана--Шеффе, такая статистика $S(T)$ единственная (очевидно, она существует, раз мы ее предъявили) и она эффективная.
\end{solution}