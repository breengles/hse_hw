%!TEX root = kotov.tex
\section*{Task:  про двух игроков}
\subsection*{Старый (новый) способ}
Пусть $A$, $B$ --- игроки; $H$, $T$ --- стороны монеты; $H_A (T_A)$ --- будет означать, что игрок $А$ выкинул решку (герб). Игра закончится, если будет два $T$ подряд.

Рассмотрим вероятность выигрыша первого игрока ($A$): по формуле полной вероятности $P(A) = \frac{1}{2} P(A|H_A) + \frac12 P(A|T_A)$. Если игрок $А$ выкинул $H$ в свой первый ход, то игра повторяется, как будто первый игрок $B$, то есть $P(A|H_A) = P(B) = 1 - P(A)$.

Разберемся с $P(A|T_A)$:
в этом случае рассмотрим эту ``ветвь'' развития игры аналогично самой игре, то есть по формуле полной вероятности $P(A|T_A) = \frac12 P(A|T_AH_B) + \frac12 P(A|T_AT_B)$. Второй член этой суммы равен нулю, так как, очевидно, что игрок $A$ не может выиграть в случае, если игрок $B$ выкинул второй $T$ подряд, следовательно, $P(A|T_AT_B) = 0 \Longrightarrow P(A|T_A) = \frac12 P(A|T_AH_B)$. Теперь рассмотрим, что происходит с игрой, в случае, когда игрок $B$ не выкинул в свой ход $T$ (то есть, как раз оставшийся вклад в $P(A|T_A)$). В таком случае $P(A|T_AH_B) = \frac12 P(A)$ (прим. игра ``начинается'' заново).
В итоге находим, что $P(A) = \frac12 (1 - P(A)) + \frac14 P(A)$.
Отсюда вычисляем $P(A) = \frac25$. Из симметричности находим $P(B) = 1 - P(A) = \frac35$

\subsection*{Числа Фибоначчи}
Рассмотрим последовательности типа $(\Omega\ldots\Omega)011$ (игра заканчивает при двух $1$ подряд).
Сначала пересчитаем, сколько существует различных комбинаций $0$ и $1$ в $(\Omega\ldots\Omega)$, не содержащих двух $1$ подряд.
Это классическая задача о расстановке двух несоседствующих перегородок ($1$ --- перегородка). У нас есть $k$ нулей, и $i$ единиц.
При условии, что единицы не могут соседствовать, что число таких расстановок ($+1$ от того, что мы можем перегородку поставить сбоку):
\begin{equation}
    \begin{pmatrix}
        k - i + 1 \\
        i
    \end{pmatrix}
\end{equation}
Всего же таких способов (надо просуммировать по $i$, с очевидным ограничением, что  единиц не должно быть больше половины, разве что для нечетной длины можно еще одну единицу вставить, иначе по Дирихле все сломается):
\begin{equation}
    \sum\limits_{i = 0}^{[k / 2] + 1}
    \begin{pmatrix}
        k - i + 1 \\
        i
    \end{pmatrix} = F_{k + 1 - 1} = F_{k} \text{ --- число Фибоначчи.}
\end{equation}
Ясно так же, что вероятность получить какую-то конкретную последовательность длины $k$ из $0$ и $1$ будет равна $\left(\frac{1}{2}\right)^{k}$. Таким образом, вероятность, что игра длины $k$ будет иметь какой-то конкретный $\Omega$-хвост, будет $\left(\frac{1}{2}\right)^{k}F_k$.

Теперь рассмотрим вероятность выйграть для первого игрока, $+1$ говорит нам, на самом деле, что первый игрок может выиграть в игре с длиной не меньше $3$ (прим. вообще говоря, хвост $011$ дает множитель $(1/2)^3$, но это как раз вписывается в эту схему, если учесть, что первый игрок может выйграть только на нечетном ходе игры, а длина игры не меньше $3$):
\begin{gather}
    P(A) = \sum\limits_{k = 1}^{\infty}\left(\frac{1}{2}\right)^{2k+1}F_{2k} = [\text{ф-ла Бине через золотое сечение}] = \\ \frac12\sum\limits_{k=1}^{\infty}\left(\frac{1}{2}\right)^{2k}\frac{\phi^{2k} - (-\phi)^{2k}}{2\phi - 1} = \frac{1}{2(2\phi - 1)}\sum\limits_{k=1}^{\infty}\left(\left(\frac{\phi}{2}\right)^{2k} - \left(\frac{-1}{2\phi}\right)^{2k}\right) = \\
    \frac{1}{2(\phi - 1)} \sum\limits_{k=1}^{\infty} \left(\frac{\phi^2}{4}\right)^k - \left(\frac{1}{4\phi}\right)^k = [\text{разбили на два ряда геом. последовательности}] = \\ \frac{2\phi^4 - 2}{(2\phi - 1)(4 - \phi^2)(4\phi^2 - 1)} = [\text{упрощаем дробь}] = \frac{2}{5}
\end{gather}
Окончательно, из симметричности $P(B) = 1 - P(A) = \frac35$