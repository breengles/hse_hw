%!TEX program = lualatex
\documentclass[a4paper,12pt]{article}
\usepackage{settings}
\usepackage{cancel}
\usepackage{ulem}
\newenvironment{task}{
    \par\noindent\textbf{Условие:}
}

\newenvironment{upd}
    {\par\noindent\textbf{Update:}}
    {\par\noindent\rule{\textwidth}{1pt}}

\newenvironment{solution}{
\begin{proof}[Решение]
    \mbox{}\par}
{\end{proof}}

\makeatletter
\newcommand\mathcircled[1]{%
  \mathpalette\@mathcircled{#1}%
}
\newcommand\@mathcircled[2]{%
  \tikz[baseline=(math.base)] \node[draw,circle,inner sep=1pt] (math) {$\m@th#1#2$};%
}
\makeatother

\setcounter{secnumdepth}{0}

\hypersetup{
    unicode=true,  % non-Latin characters in Acrobat’s bookmarks
    pdffitwindow=true,  % window fit to page when opened
    pdfstartview={FitH},  % fits the width of the page to the window
    pdfauthor={Котов А. А.},  % author
    pdfnewwindow=true,  % links in new PDF window
    colorlinks=true,  % false: boxed links; true: colored links
    linkcolor=linkcolor,  % color of internal links (change box color with linkbordercolor)
    citecolor=citecolor,  % color of links to bibliography
    filecolor=magenta,  % color of file links
    urlcolor=urlcolor  % color of external links
}

\definecolor{dkgreen}{rgb}{0,0.6,0}
\definecolor{gray}{rgb}{0.5,0.5,0.5}
\definecolor{mauve}{rgb}{0.58,0,0.82}  

\lstset{ %
language=python,                % the language of the code
basicstyle=\footnotesize,           % the size of the fonts that are used for the code
numbers=left,                   % where to put the line-numbers
numberstyle=\tiny\color{gray},  % the style that is used for the line-numbers
stepnumber=1,                   % the step between two line-numbers. If it's 1, each line 
                                % will be numbered
numbersep=5pt,                  % how far the line-numbers are from the code
backgroundcolor=\color{white},      % choose the background color. You must add \usepackage{color}
showspaces=false,               % show spaces adding particular underscores
showstringspaces=false,         % underline spaces within strings
showtabs=true,                 % show tabs within strings adding particular underscores
frame=single,                   % adds a frame around the code
rulecolor=\color{black!10},        % if not set, the frame-color may be changed on line-breaks within not-black text (e.g. comments (green here))
tabsize=4,                      % sets default tabsize to 2 spaces
captionpos=b,                   % sets the caption-position to bottom
breaklines=true,                % sets automatic line breaking
breakatwhitespace=false,        % sets if automatic breaks should only happen at whitespace
title=\lstname,                   % show the filename of files included with \lstinputlisting;
                                % also try caption instead of title
keywordstyle=\color{blue},          % keyword style
commentstyle=\color{dkgreen},       % comment style
stringstyle=\color{mauve},        % string literal style
escapeinside={\%*}{*)},            % if you want to add LaTeX within your code
morekeywords={done, to},              % if you want to add more keywords to the set
%  deletekeywords={...}              % if you want to delete keywords from the given language
}


\author{Котов Артем, МОиАД ВШЭ СПб}
\title{Практическая работа №1}
\date{\today}

\begin{document}
\maketitle
\tableofcontents
\newpage


\no Параметры:
\begin{itemize}
    \item $a \in [75, 90]$
    \item $b \in [500, 600]$
    \item $c \in [0, 690]$
    \item $d \in [1380]$
    \item $p_1 = 0.1$
    \item $p_2 = 0.01$
    \item $p_3 = 0.3$
\end{itemize}

\no Обозначения:
\begin{itemize}
    \item Обычно, в суммах не указаны нижние и верхние пределы, что означает, что суммирование ведется по всем возможным значениям соответствующих величин.
    \item $\Bin$ --- биномиальное распределение.
    \item $\Poisson$ --- Пуассоновское распределение.
\end{itemize}



%!TEX root = kotov.tex
\section{Task 1: Второе по минимальности MST}
\begin{task}
    Найдите за $\O(V^2 + E)$ второе по весу остовное дерево в неографе.
\end{task}

\begin{solution}
    Давайте запускать Прима, но в процессе которого будем запоминать самое тяжелое ребро между парами любых ребер, например, мы можем позволить себе при добавлении нового ребра делать следующее: пусть оно соединяет вершины $v$ и $u$, где $v \in S$, а $u \in \bar{S}$, где $S$ текущее дерево.
    Пересчитаем для всех $v'\in S$ следующий массивчик $d[v'][u] = max(d[v'][v], (vu))$, то есть похоже на какую-то простенькую динамику на каждом шаге добавления новой вершины $u$ к вершине $v$.
    Все равно у нас это у нас мажорировано самим поиском MST, поэтому такая добавочка во внешний цикл \texttt{while not q.empty()} (здесь $q$ --- очередь с приоритетом на массиве для классического Прима) не попортит сложности.
    
    \begin{upd}
        Рассмотрим как Прим добавляет вершины: начинаем с какой-то вершины, пусть $v$, добавляем в дерево эту одну вершину (т.е. сейчас $S = v$, а $\bar{S} = G \smallsetminus S = G \smallsetminus v$), затем рассматриваем все ребра, которые соединяют вершину $v$ с вершинами из $\bar{S}$, выбираем минимальное по весу ребро (пусть $(vu)$) и добавляем эту вершинку и это ребро в наше искомое MST.


        Но теперь есть дополнительное действие, которое мы делаем, у нас есть массив, который хранит самое тяжелое ребро на пути $v \rightarrow u$: $d[v][u]$.
        То есть в этом примере на этом шаге мы бы просто записали $d[v][u] = w_{(vu)}$, на следующих итерациях классического Прима мы присоединяем какую-то вершину $u \in \bar{S}$ к какой-то вершине $v \in S$, при этом нам надо записать максимальное ребро на пути от каждой вершины, которые уже есть в дерево, до новодобавленной, но при этом мы уже знаем самое тяжелое ребро на пути от каждой вершины дерева до вершины $v$, то есть до той, к которой мы добавляем (это мы получили на каком-то из предыдущих шагов итерации).

        Для этого мы просто пробегаемся по всем вершинами из текущего дерева, то есть по всем $v' \in S$ и смотрим на самое тяжелое ребро на пути $v' \rightarrow u$, причем для этого (так как дерево) достаточно сравнить самое тяжелое ребро на пути $v' \rightarrow v$ с новодобавленным ребром и выбрать из них наибольшее. Ну и да, правильнее сказать, что мы храним пару (вес, ребро) в этом массиве.

        Зачем нам вообще этот массив? Чтобы быстро можно было отвечать на вопрос ``какое самое тяжелое ребро на пути в MST'', потому что последующая процедура поиска второго MST как раз будет разрывать искусственно сделанный цикл по этому ребру, потому что если бы мы разорвали не по этому, то мы бы получили заведомо более тяжелое дерево.
    \end{upd}

    Теперь, когда у нас есть в руках MST и такой массив самых тяжелых ребер между вершинами будем делать следующее: пробежимся по всем ребрам, которые не принадлежат построенному MST, их всего $E - V + 1$ штука, для каждого ребра $(uv) \notin T$ будем добавлять такое ребро к $T$ и удаляться самое тяжелое ребро на пути $u \leftarrow v$ (это мы делаем очень быстро за счет предподсчитанного массива самых тяжелых ребер), берем среди всех таких новых деревьев минимальное по весу. По сложности у нас был улучшенный Прим за $\O(V^2)$, еще у нас была пробежка по всем ребрам не из MST $\O(E - V)$, в итоге будет $\O(V^2 + E)$.

\end{solution}
%!TEX root = kotov.tex
\section{Task 2: Почти равенство}
\begin{solution}
    \begin{enumerate}[a)]
        \item 
    \end{enumerate}    
\end{solution}
%! TEX root = kotov.tex

\section{Task 3: Pareto as exp.}

\begin{task}
    Записать распределение Парето с плотностью $p(x|\alpha, \beta) = \frac{\alpha \beta^{\alpha}}{x^{\alpha + 1}}[x \ge \beta]$ при фиксированном $\beta$ в форме экспоненциального класса распределений. Найти $\E \ln x$ путём дифференцирования нормировочной константы.
\end{task}

\begin{solution}

    Приведем распределение Парето к экспоненциальному виду:
    \begin{gather}
        p(x|\alpha, \beta) = \frac{\alpha \beta^{\alpha}}{x^{\alpha + 1}}[x \ge \beta], \quad \beta \text{ --- fixed.} \\
        p(x | \alpha) = \frac{f(x)}{g(\alpha)}\exp(\alpha u(x)) \\
        p(x | \alpha, \beta) = \frac{\alpha \beta^{\alpha}}{x} \exp(-\alpha \ln x)[x \ge \beta] \\
        \Downarrow \\
        f(x) = \frac{[x \ge \beta]}{x}; \quad g(\alpha) = \frac{1}{\alpha \beta^{\alpha}}; \quad u(x) = -\ln x
    \end{gather}

    Теперь разберемся с матожиданием:
    \begin{gather}
        \E \ln x = \frac{\partial g(\alpha)}{\partial \alpha} = -\frac{1}{\alpha^2 \beta^{2 \alpha}}\left( \beta^{\alpha} + \alpha \beta^{\alpha} \ln \beta \right) = - \frac{1 + \alpha \ln \beta}{\alpha^2 \beta}
    \end{gather}
\end{solution}
%!TEX root = kotov.pdf
\section{Task 4}
\begin{task}
    Про антиклики,
\end{task}

\begin{solution}
    \begin{enumerate}[a)]
        \item Этот пункт будет сделан в стиле древнего способа деления числе: то есть что такое целая часть $a/b$? Это то, сколько раз (минус один) мы можем вычесть из $a$ число $b$ пока результат не станет $\leq 0$.
        Поступим в этой задаче самым наивным образом, а именно возьмем первую вершину, ``удалим'' саму вершину (но возьмем ее в антиклику) и всех ее соседей из графа, то есть из условия задачи мы удалим не более чем $d+1$ вершин, перейдем к следующей вершине и поступим таким же образом, и т.д. В итоге мы сделаем не менее $\frac{n}{d+1}$ действие, на котором будем брать по одной вершине, то есть в итоге в нашей антиклике будет содержаться как минимум $\frac{n}{d+1}$ вершин (то есть это старинный способ деления, только нашем случае нам можно не вычитать единицу из ``целой части'', так как хотя бы одну вершину антиклика все-таки содержит).
        
        По сложности мы должны будем пройтись один разочек по исходному массиву, для каждой вершины делая в худшем случае $d+1$ действие, связанное, например, с тем, что мы запоминаем какие вершины уже удалены (что-то типа массива $0$ и $1$ для исходного массива, где изначально все элементы $1$, а $0$ означает, что мы удалили вершину из множества). В итоге сложность будет $\O((d+1)\frac{n}{d+1}) = \O(n)$ в худшем случае.
        \item *грустный смайлик.png*
    \end{enumerate}
\end{solution}
%!TEX root = kotov.tex
\section{Task 1.1 (Случайный доп.)}
\begin{task}
    У каждой вершины не более $3$ врагов. Вражда – симметричное отношение. Разбить вершины на $2$ доли так, чтобы с вершиной в долю попало не более $1$ врага. $\O(V + E)$.
\end{task}

\begin{solution}
    Честно, я случайно начал ее решать, но раз уже есть какое-то решение, то приведу его.

    Представим эти отношения враждебности как ребра на графе, т.е. из условия следует, что степень каждой вершины не больше $3$.
    Заведем два цвета для двух долей: красный и синий.
    И будем отслеживать активный цвет, например, на старте красный.
    Запустим поиск в глубину и будем красить вершины в активный цвет. Если наткнемся на ситуацию, в которой число смежных вершин, покрашенных в активный цвет больше $1$, то меняем активный цвет на другой.

    Почему это, вероятно, работает?
    Рассматриваем только связные графы, иначе можно было бы разбить задачу на несколько подзадач на каждой из компонент связности. Поиском в глубину можно построить своего рода подвешенное дерево на вершине, из которой запускаемся. Если такое дерево содержало бы вообще все ребра исходного графа, то было бы все просто, так как мы бы просто красили ветки в разные (чередующиеся) цвета (слова передаются с трудом). Но такое дерево не видит ребер, которые могут идти в одной ветви из одной вершины в другую (на семинаре уже ``показали'', что ребер между ветками быть не может). Тогда если бы у нас была ветка одного цвета, то мы бы покрасили ее неправильно, ну как раз для этого мы в какой-то момент меняли цвет раскраски, чтобы такие вершины были разного цвета. Так как у нас максимум степень вершины $3$, то у нас как раз хватит двух цветов, чтобы покрасить ее и ее соседей так, чтобы не было больше одной одноцветной пары для одной и той же вершины (опять сложное перекладывание мысли в слова).
\end{solution}
%!TEX root = kotov.tex
\section{Task 6}
Рассмотрим исходный массив $[a_1, \ldots, a_n]$
\begin{enumerate}
    \item Создадим новый массив $[b_1, \ldots, b_n]$ такой, что $b_1 = a_1$, $b_i = a_i - a_{i-1}$ для $i=2,\ldots,n$, то есть это разница между соседними элементами исходного массива, сложность $O(n)$.
    \item Для каждого запроса $\text{add}(l,r,x)$ сделаем следующее:
    \begin{itemize}
        \item $b_l\,\,+\!\!= x$.
        \item Если $r<n$, то $b_{r+1} \,\, -\!\!= x$.
    \end{itemize}
    \begin{remark}
        Обработка каждого запроса делает за константное время $O(1)$, тогда для $m$ запросов будет $O(m)$
    \end{remark}
    \item После обработки запросов будем выводить исходный массив по следующей схеме: $a_i = a_{i-1} + b_i$, а вот эта штука уже делается для каждого $i=1,\ldots,n \Longrightarrow$ сложность $O(n)$ для вывода. Результирующая сложность $O(n + m)$
    \begin{remark}
        Почему $a_i$ выведенная таким образом будет является корректным? Рассмотрим, для простоты, один запрос $\text{add}(l,r,x)$.
        
        Для членов с $i<l a_i = a_{i-1} + b_i = a_{i-1} + a_i - a_{i-1} = a_i$, то есть начало массива не изменилось, как и должно быть.
        
        Для $i = l:a_l = a_{l-1} + a_{l} - a_{l-1} + x = a_l + x$, то есть начало поданного отрезка действительно увеличилось на $x$, для последующих элементов $i=l,\ldots,r$ $x$ будет уже содержаться в предыдущем элементе, а $b_i$ не содержат добавочки в виде $x$, но будет компенсировать изначальное (не увеличенное на $x$) значение.
        
        Для $i=r+1:b_{r+1} = a^{\text{old}}_{r+1} - a^{\text{old}}_{r} - x$, то есть для $a_{r+1} = a_r + b_{r+1} = a_r + a^{\text{old}}_{r+1} - a^{\text{old}}_r - x = a^{\text{old}}_{r+1} + \underbrace{(a_r - a^{\text{old}}_r)}_{x} - x = a^{\text{old}}_{r+1}$, таким образом, мы подавили вклад $x$ для $a_{r+1}$ элемента, то есть элемент не изменился, и последующие элементы не содержат $x$.
    \end{remark}
\end{enumerate}

\end{document}