%! TEX root = kotov.tex

\section{Task 4}
\begin{task}
    \begin{equation}
        \frac{\partial Tr(AX^{-T}BXC)}{\partial X} = ?
    \end{equation}
\end{task}


\begin{solution}
    Сначала определимся с размерностями, чтобы ничего не сломать: пусть $X \in \mathbb{R}^{n x n}$, а $C \in \mathbb{R}^{n x m}$, тогда $B \in \mathbb{R}^{n x n}$, $A \in \mathbb{R}^{m x n}$.

    Опять будем делать через дифференциал:
    \begin{equation}
        Tr(AX^{-T}BXC) = Tr(CAX^{-T}BX) = \braket{A^TC^T | X^{-T}BX}
    \end{equation}
    \begin{gather}
        d(\braket{A^TC^T | X^{-T}BX}) = \braket{A^TC^T | d(X^{-T}BX)} = (*)
    \end{gather}

    \begin{enumerate}
        \item $d(X^{-T}BX) = d(X^{-T}B)X + X^{-T}BdX$
        \item $d(X^{-T} B) = (-X^{-1} dX X^{-1})^T B$
    \end{enumerate}

    \begin{gather}
        (*) = (-X^{-1} dX X^{-1})^T B X + X^{-T} B dX \\
        = -X^{-T} dX^T X^{-T} B X + X^{-T} B dX \\
        = X^{-T} (-dX^T X^{-T} B X + (dX^T B^T)^T) \\
        = -X^{-T}((X^T B^T X^{-1} dX)^T - B dX)
    \end{gather}

    Т.о. получаем:
    \begin{gather}
        \braket{A^T C^T | -X^{-T}((X^T B^T X^{-1} dX)^T - B dX)} \\
        = \braket{A^T C^T | -X^{-T} (X^T B^T X^{-1} dX)^T} + \braket{A^T C^T | X^{-T} B dX} =
    \end{gather}
    перекидываем от $dX$ все, собираем скалярное произведение обратно в кучу (честно, не очень хочется это набирать в техе):
    \begin{gather}
        = \braket{B^T X^{-1} A^T C^T - X^{-T} B X C A X^{-T} | dX} \\
        \Downarrow \\
        \frac{\partial Tr(AX^{-T}BXC)}{\partial X} = \underbrace{B^T X^{-1} A^T C^T}_{\mathcircled{1}} - \underbrace{X^{-T} B X C A X^{-T}}_{\mathcircled{2}}
    \end{gather}

    Проверим, что размерности сошлись:
    \begin{enumerate}
        \item $(n, n)(n, n)(n, m)(m, n) \Longrightarrow (n, n)$
        \item $(n, n)(n, n)(n, n)(n, m)(m, n)(n, n) \Longrightarrow (n, n)$
    \end{enumerate}
    В итоге, получим подходящую размерность, т.к. дифференцировали скаляр по матрице и ожидали получить матрицу размера $(n, n)$.
\end{solution}
